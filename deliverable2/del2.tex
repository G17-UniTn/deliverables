\documentclass[11pt, a4paper]{article}

%\usepackage[T1]{fontenc}
%\usepackage{fullpage}

\usepackage[utf8]{inputenc} % comment when using lualatex
\usepackage[italian]{babel} % lingua e a-capo-sillabato
\usepackage{graphicx}
\usepackage[hidelinks]{hyperref} % link di pagina
\usepackage[bottom]{footmisc} % note appiccicate al fondo della pagina
\usepackage{float} % per posizionamento immagini
\usepackage{amsthm} % per ambienti stile teorema
\usepackage{tabularx} %tabelle
\usepackage[table]{xcolor} %colore caselle
\usepackage{enumitem} %additional commands for lists
\usepackage{fancyhdr}



\pagestyle{fancy}
\fancyhf{}% Clear header/footer
\fancyhead[C]{\footnotesize\textit{Documento:} D2 \hfill SleepCode \hfill \textit{Versione:} 1.0}
\renewcommand{\headrule}{{\color{red!70}\rule{\textwidth}{2pt}}}
\setlength{\headheight}{22pt}

\renewcommand\UrlFont{\color{blue}\rmfamily} % colore link

\theoremstyle{definition} % stile dei newtheorem (non italizzati)
\newtheorem{funcreq}{RF} %% numerazione dei requisiti funzionali
\newtheorem{nonfuncreq}{RNF} %% requisiti non funzionali
\newtheorem{backend}{BE}
\newtheorem{frontend}{FE}




\title{Specifica dei Requisiti}

\author{Raffaele \textsc{Castagna}\\
Alberto \textsc{Rovesti}\\
Zeno \textsc{Saletti}}

\newcommand{\groupNumber}{G17}

% Web address for the project (if any)
% \newcommand{\homepage}{\url{https://www.}}

% data
\date{\today}

\makeatletter{}

% IL PREAMBOLO FINISCE QUI %%%%%%%%%%%%%%%%%%%%%%%%%%%%%%%%%%%%%%%%%%%%%%%%%%%%



\begin{document}

% La pagina di copertina si trova in un file .tex a parte
% NON MODIFICARE QUESTO COMANDO!!!
\begin{titlepage}
\newcommand{\HRule}{\rule{\linewidth}{0.3mm}} % Defines a new command for horizontal lines, change thickness here
\center % Centre everything on the page

%------------------------------------------------
%	Logo
%------------------------------------------------
\includegraphics[width=0.3\textwidth]{materiale/UniTrento_logo_ITA_colore.png}\\[0.5cm]
%------------------------------------------------
%	Headings
%------------------------------------------------
\textsc{\Large Dipartimento di Ingegneria\\e Scienza dell'Informazione}\\[1.5cm]

{\Huge\textbf{Sleep Code}}\\[0.5cm]
\textsc{\large Progetto per il Corso di Ingegneria del Software}\\
\textsc{\large Anno Accademico 2023-2024}\\[0.5cm]

%------------------------------------------------
%	Title
%------------------------------------------------

\HRule\\[0.4cm]
{\huge\bfseries \@title}\\[0.1cm]
\HRule\\[1cm]

\begin{minipage}{\textwidth}
\begin{flushleft}
\textit{Descrizione:} documento di analisi dei requisiti funzionali, non funzionali, front-end e back-end.
\end{flushleft}
\end{minipage}\\[1.5cm]


\begin{minipage}{0.4\textwidth}
\begin{flushleft}
\large
\textit{Numero documento:} D1
\end{flushleft}
\end{minipage}
\begin{minipage}{0.4\textwidth}
\begin{flushright}
\large
\textit{Versione documento:} 2.4
\end{flushright}
\end{minipage}\\[1.5cm]

%------------------------------------------------
%	Author(s)
%------------------------------------------------
\begin{minipage}{0.4\textwidth}
\begin{flushleft}
\large
\textit{Membri del gruppo:}\\
\@author % Your name
\end{flushleft}
\end{minipage}
~
\begin{minipage}{0.4\textwidth}
\begin{flushright}
\large
\textit{Numero gruppo: }
\groupNumber
\end{flushright}
\end{minipage}

% 	If you don't want a supervisor, uncomment the two lines below and comment the code above
% 	{\large\textit{Author(s)}}\\
% 	\@author % Your name

%------------------------------------------------
%	Date
%------------------------------------------------

\vfill\vfill
\textit{Ultima revisione:}
{\@date}

\end{titlepage}

\tableofcontents

\newpage

\section*{Scopo del documento}
Il presente documento riporta la specifica dei requisiti di sistema
del progetto SleepCode ricorrendo a diagrammi realizzati secono lo
Unified Modeling Language (UML) e tabelle.

\section{Requisiti funzionali}
In questa sezione vengono descritti i requisiti funzionali (RF) del
servizio utilizzando alcuni Use Case Diagrams (UCD) scritti in UML,
eventualmente arricchiti da descrizioni in linguaggio naturale.


\subsection{Accesso}
\begin{itemize}
    \item \textbf{RF 1.} Registrazione
    \item \textbf{RF 2.} Login
    \item \textbf{RF 3.} Recupero password
\end{itemize}

\begin{figure}[H]
\centering
\includegraphics[scale=0.65]{materiale/accesso.pdf}
\caption{UCD dello scenario di accesso al servizio}
\end{figure}

\hrule
\subsubsection*{Descrizione Use Case \textit{Registrare un nuovo account}}
\begin{description}
    \item[Titolo:] Registrazione account
    
    \item[Riassunto:] Questo Use Case descrive come l'utente anonimo deve
    effettuare la registrazione sulla piattaforma.

    \item[Descrizione:]
    \begin{enumerate}
        \item[]
        \item L'utente anonimo accede alla pagina dedicata e sceglie tra la registrazione mediante il sistema di credenziali interno oppure per mezzo di un account Google.
        \item La registrazione con sistema di credenziali interno prevede l'inserimento di un username, un'email valida e una password conforme (RNF). \verb|[exception 1]|\verb|[exception 2]|\verb|[exception 3]|
        \item La password deve essere confermata reinserendola in un secondo campo.
        \item L'account viene registrato dal servizio di database.
    \end{enumerate}
    
    \item[Exceptions:]
    \begin{itemize}
        \item[]
        \item \verb|[exception 1]| Se l'email inserita è malformata, inesistente o associata ad un account già registrato, la registrazione non può proseguire e l'utente viene avvisato.
        \item \verb|[exception 2]| Se la password non è conforme, la registrazione non può proseguire e l'utente viene avvisato.
        \item \verb|[exception 3]| Se i campi di inserimento e di conferma della password contengono stringhe che non coincidono, la registrazione non può proseguire e l'utente viene avvisato.
    \end{itemize}
\end{description}

\subsubsection*{Descrizione Use Case \textit{Recuperare la password}}
\begin{description}
    \item[Titolo:] Recupero password
    
    \item[Riassunto:] Questo Use Case descrive come l'utente anonimo e
    registrato alla piattaforma, facendo affidamento al sistema di credenziali
    interno, può recuperare il proprio account qualora la password venisse dimenticata.

    \item[Descrizione:]
    \begin{enumerate}
        \item[]
        \item L'utente accede alla pagina di recupero, per mezzo di quella di login.
        \item La pagina indica all'utente di inserire l'email di recupero, ovvero quella associata all'account. \verb|[exception 1]|
        \item Il sistema richiede al servizio di posta elettronica l'invio di un link di recupero mediane un messaggio email, specificando l'indirizzo fornito dall'utente.
        \item Il link guida l'utente dal messaggio alla pagina del sistema dedicata alla creazione di una nuova password.
        \item L'utente inserisce una nuova password conforme e la conferma, inserendola nuovamente. \verb|[exception 2]|\verb|[exception 3]|
        \item Il servizio di database provvede all'aggiornamento della password.
    \end{enumerate}
    
    \item[Exceptions:]
    \begin{itemize}
        \item[]
        \item \verb|[exception 1]| Se l'email inserita è malformata, inesistente o associata ad un account non registrato, il recupero non può proseguire e l'utente viene avvisato.
        \item \verb|[exception 2]| Se la password non è conforme, la registrazione non può proseguire e l'utente viene avvisato.
        \item \verb|[exception 3]| Se i campi di inserimento e di conferma della password contengono stringhe che non coincidono, la registrazione non può proseguire e l'utente viene avvisato.
    \end{itemize}
    
    %\item[Extensions:]
\end{description}


\newpage
\subsection{Consultazione dei problemi}
\begin{itemize}
    \item \textbf{RF 4.} Consultazione del catalogo dei problemi
    \item \textbf{RF 5.} Consultazione di un problema
    \item \textbf{RF 8.} Metadati aggiuntivi
    \item \textbf{RF 9.3.} Progressi
\end{itemize}
\subsection*{Modifica del catalogo dei problemi}
\begin{itemize}
    \item \textbf{RF 12.} Aggiungere un problema
    \item \textbf{RF 13.} Modificare un problema
    \item \textbf{RF 14.} Eliminare un problema
\end{itemize}

\begin{figure}[H]
\centering
\includegraphics[scale=0.55]{materiale/consulta-catalogo.pdf}
\caption{UCD dello scenario della consultazione dei problemi e della modifica del catalogo}
\end{figure}
\hrule

\subsubsection*{Descrizione Use Case \textit{Aggiungere problemi}}
\begin{description}
    \item[Titolo:] Aggiungere un problema
    
    \item[Riassunto:] Questo Use Case descrive come l'utente amministratore
    deve aggiungere nuovi problemi al catalogo.

    \item[Descrizione:]
    \begin{enumerate}
        \item[]
        \item L'utente amministratore accede alla pagina del catalogo e sceglie di aggiungere un nuovo problema.
        \item L'utente compila i campi necessari alla creazione di un nuovo problema: i dati relativi alla struttura, quali titolo, descrizione e almeno tre esempi di input e output atteso; dati descrittivi, ovvero nome, selezione della difficoltà (bassa, intermedia, alta), categoria e link al video-suggerimento.
        \item L'utente conferma la creazione del problema, che viene quindi aggiunto al catalogo; alternativamente, l'utente può scegliere di annullare la creazione del nuovo problema, previo avviso e conferma da parte del sistema.\texttt{[exception 1]}\texttt{[exception 2]}
    \end{enumerate}
    
    \item[Exceptions:]
    \begin{itemize}
        \item[]
        \item \verb|[exception 1]| Se tra i dati strutturali del problema è presente almeno un campo non compilato, l'aggiunta del problema al catalogo non viene eseguita e l'utente viene avvisato.
        \item \verb|[exception 2]| Se è presente almeno un campo vuoto tra quelli destinati ai dati descrittivi del problema, l'operazione non può essere completata e l'utente viene avvisato.
    \end{itemize}
\end{description}

\subsubsection*{Descrizione Use Case \textit{Modificare problemi}}
\begin{description}
    \item[Titolo:] Modificare un problema
    
    \item[Riassunto:] Questo Use Case descrive come l'utente amministratore
    deve modificare i problemi.

    \item[Descrizione:]
    \begin{enumerate}
        \item[]
        \item L'utente amministratore accede alla pagina del catalogo e seleziona un problema da modificare.
        \item L'utente modifica i campi strutturali (titolo, descrizione, esempi di input e output) e descrittivi (nome, difficoltà, categoria e link al video-suggerimento) del problema.
        \item L'utente conferma la modifica del problema, che verrà poi aggiornato nel catalogo, oppure conferma di annullare la modifica.\texttt{[exception 1]}
    \end{enumerate}
    
    \item[Exceptions:]
    \begin{itemize}
        \item[]
        \item \verb|[exception 1]| Se tra i dati strutturali del problema è presente almeno un campo non compilato, l'aggiunta del problema al catalogo non viene eseguita e l'utente viene avvisato.
        \item \verb|[exception 2]| Se è presente almeno un campo vuoto tra quelli destinati ai dati descrittivi del problema, l'operazione non può essere completata e l'utente viene avvisato.
    \end{itemize}

    \item[Extensions:]
    \begin{itemize}
        \item[]
        \item \verb|[extension 1]|
    \end{itemize}
\end{description}

\newpage
\subsection{Esercitazione}
\begin{itemize}
    \item \textbf{RF 6.} Avviare l'esercitazione
    \item \textbf{RF 7.} Correttezza dell'algoritmo
    \item \textbf{RF 9.1.} Registrare i progressi
\end{itemize}

\begin{figure}[H]
\centering
\includegraphics[scale=0.57]{materiale/esercitazione.pdf}
\end{figure}

\subsubsection*{Descrizione Use Case \textit{Registrare i progressi}}

\newpage
\subsection{Gestione del profilo e dell'account}
\begin{itemize}
    \item \textbf{RF 9.2.} Progressi
    \item \textbf{RF 10.} Aggiornamento dei dati dell'account
    \item \textbf{RF 11.} Logout
\end{itemize}

\begin{figure}[H]
\centering
\includegraphics[scale=0.8]{materiale/gestione-account.pdf}
\caption{UCD dello scenario}
\end{figure}

\newpage
\section{Requisiti non funzionali}
Nella presente sezione sono riportati i requisiti non funzionali (RNF)
del sistema utilizzando tabelle strutturate e specificando misure che
consentano di effettuare facilmente delle verifiche quantitative.

\subsection{Caratteristiche di sistema}

\begin{nonfuncreq}
\textbf{Scalabilità }
\begin{center}
    \footnotesize
    \begin{tabularx}{\textwidth}{|X||X||X|}
        \hline
        \cellcolor{red!70}Proprietà & \cellcolor{red!70}Descrizione & \cellcolor{red!70}Misura\\
        \hline
        Elaborazione con un numero crescente di utenti. & Capacità del sistema di gestire un numero crescente di utenti in simultanea. & Viene garantito l'accesso in simultanea di almeno 300 utenti nel primo anno dal lancio.\\
        \hline
        Memorizzazione dei dati degli utenti & Capacità del sistema di gestire i dati generati da un numero crescente di utenti utilizzatori. & Capacità sufficiente per almeno 400 utenti. \\
        \hline
    \end{tabularx}
\end{center}
\end{nonfuncreq}

\begin{nonfuncreq}
    \textbf{Compatibilità }
    \begin{center}
        \footnotesize
        \begin{tabularx}{\textwidth}{|X||X||X|}
            \hline
            \cellcolor{red!70}Proprietà & \cellcolor{red!70}Descrizione & \cellcolor{red!70}Misura\\
            \hline
            Compatibilità client & La piattaforma del servizio deve essere compatibile con e accessibile attraverso le versioni più recenti dei principali browser in commercio. &
            \begin{itemize}
                \item Chrome
                
                117.0.5938.150
                \item Firefox
                
                18.0.1
                \item Edge:
                
                17.0.2045.60
            \end{itemize}La compatibilità deve valere anche per le rispettive versioni superiori.\\
            \hline
        \end{tabularx}
    \end{center}
\end{nonfuncreq}

\begin{nonfuncreq}
    \textbf{Usabilità }
    \begin{center}
        \footnotesize
        \begin{tabularx}{\textwidth}{|X||X||X|}
            \hline
            \cellcolor{red!70}Proprietà & \cellcolor{red!70}Descrizione & \cellcolor{red!70}Misura\\
            \hline
            Usabilità & Intuitività e facilità nell'apprendimento, accesso e impiego delle funzionalità fornite dal servizio. & Il nuovo utente deve poter conoscere e utilizzare il 90\% delle funzionalità (disponibili al proprio livello di accesso) in meno di 30 minuti.\\
            \hline
        \end{tabularx}
    \end{center}
\end{nonfuncreq}

\newpage
\begin{nonfuncreq}
    \textbf{Aspetto }
    \begin{center}
        \footnotesize
        \begin{tabularx}{\textwidth}{|X||X||X|}
            \hline
            \cellcolor{red!70}Proprietà & \cellcolor{red!70}Descrizione & \cellcolor{red!70}Misura\\
            \hline
            Colore & Gamma cromatica dell'interfaccia e distribuzione del colore. La scelta ricade su colori, tinte (aggiunta di bianco) e sfumature (aggiunta di nero) che mirano a limitare l'affaticamento della vista. & Colori caldi; colori freddi presenti in sfumature scure; colori freddi accesi presenti al più in aree ristrette (pulsanti e icone).\\
            \hline
            Contrasto & Accostamento dei colori all'interno dell'interfaccia utente. Mira alla leggibilità e alla limitazione dell'affaticamento della vista. & Regola dei complementari; cerchio di Itten.\\
            \hline
        \end{tabularx}
    \end{center}
\end{nonfuncreq}

\begin{nonfuncreq}
    \textbf{Lingua }
    \begin{center}
        \footnotesize
        \begin{tabularx}{\textwidth}{|X||X||X|}
            \hline
            \cellcolor{red!70}Proprietà & \cellcolor{red!70}Descrizione & \cellcolor{red!70}Misura\\
            \hline
            Lingua di sistema           & Lingua presente nell'interfaccia e nelle risorse fornite dal servizio. & L'interfaccia generale della piattaforma contiene testo in italiano (100\%); i testi dei problemi sono scritti in italiano (100\%); le risorse multimediali (video-suggerimento) devono essere in italiano oppure in inglese.\\
            \hline
        \end{tabularx}
    \end{center}
\end{nonfuncreq}

\begin{nonfuncreq}
    \textbf{Prestazioni }
    \begin{center}
        \footnotesize
        \begin{tabularx}{\textwidth}{|X||X||X|}
            \hline
            \cellcolor{red!70}Proprietà & \cellcolor{red!70}Descrizione & \cellcolor{red!70}Misura\\
            \hline
            Caricamento all'accesso & Tempo massimo richiesto per caricare le pagine rilevanti dopo la ricerca in browser. & Il caricamento delle pagine di login e home (per quest'ultima si considera l'intervallo di tempo che comincia dopo la richiesta di autenticazione) non deve eccedere i 2 secondi.\\
            \hline
            Transizioni & Tempo massimo richiesto per effettuare una transizione da una pagina all'altra.  & Una transizione non deve richiedere più di 2 secondi.\\
            \hline
        \end{tabularx}
    \end{center}
\end{nonfuncreq}


\subsection{Affidabilità}

\begin{nonfuncreq}
    \textbf{Downtime }
    \begin{center}
        \footnotesize
        \begin{tabularx}{\textwidth}{|X||X||X|}
            \hline
            \cellcolor{red!70}Proprietà & \cellcolor{red!70}Descrizione & \cellcolor{red!70}Misura\\
            \hline
            Downtime & Tempo medio massimo in cui il servizio non è raggiungibile; principalmente per motivi di manutenzione e aggiornamento. & 2,7\% (240 ore) nel primo anno 0,85\% (72 ore) dopo il primo anno dal lancio.\\
            \hline
        \end{tabularx}
    \end{center}
\end{nonfuncreq}

\begin{nonfuncreq}
    \textbf{Disponibilità }
    \begin{center}
        \footnotesize
        \begin{tabularx}{\textwidth}{|X||X||X|}
            \hline
            \cellcolor{red!70}Proprietà & \cellcolor{red!70}Descrizione & \cellcolor{red!70}Misura\\
            \hline
            Disponibilità & Probabilità che il sito non si guasti entro un intervallo di tempo trascorso dopo l'entrata in servizio. & Probabilità di resistere ai guasti al 98\% entro le prime 8.000 ore.\\
            \hline
        \end{tabularx}
    \end{center}
\end{nonfuncreq}

\subsection{Privacy e sicurezza}

\begin{nonfuncreq}
    \textbf{Privacy e trattamento dei dati }
    \begin{center}
        \footnotesize
        \begin{tabularx}{\textwidth}{|X||X||X|}
            \hline
            \cellcolor{red!70}Proprietà & \cellcolor{red!70}Descrizione & \cellcolor{red!70}Misura\\
            \hline
            Normativa & Conformità con le vigenti norme relative al trattamento e alla protezione dei dati (GDPR). In particolare, i dati personali dell'utente registrato (nome, email e password) non devono essere divulgati in alcun modo e, qualora lo ritenga opportuno, l'utente ha il diritto di richiedere l'eliminazione delle proprie informazioni dal servizio al fine di interrompere il trattamento. & Conformità del servizio e funzionalità a supporto dell'utente (eliminazione account).\\
            \hline
        \end{tabularx}
    \end{center}
\end{nonfuncreq}

\begin{nonfuncreq}
    \textbf{Connessione sicura }
    \begin{center}
        \footnotesize
        \begin{tabularx}{\textwidth}{|X||X||X|}
            \hline
            \cellcolor{red!70}Proprietà & \cellcolor{red!70}Descrizione & \cellcolor{red!70}Misura\\
            \hline
            Connessione sicura & Impiego di protocolli di comunicazione che garantiscono la confidenzialità e riservatezza delle informazioni scambiate tra client e server. & Utilizzo del protocollo \texttt{https}.\\
            \hline
        \end{tabularx}
    \end{center}
\end{nonfuncreq}

\begin{nonfuncreq}
    \textbf{Password strength }
    \begin{center}
        \footnotesize
        \begin{tabularx}{\textwidth}{|X||X||X|}
            \hline
            \cellcolor{red!70}Proprietà & \cellcolor{red!70}Descrizione & \cellcolor{red!70}Misura\\
            \hline
            Password sicura & Quantità e varietà di caratteri necessari per comporre una password forte. & Una password conforme possiede da 8 a 64 caratteri, tra i quali sono presenti almeno: una lettera maiuscola, una minuscola, una cifra decimale e un carattere speciale tra ! ? \# \$ \% \& @ * + - / $\backslash$ = \_ . , ; : ( ) [ ] \{ \}.\\
            \hline
        \end{tabularx}
    \end{center}
\end{nonfuncreq}






\newpage
\section{Analisi del contesto}
% riguarda il backend e i componenti esterni al sistema: tutto il codice che utilizzate ma che non avete scritto voi
% flusso di informazioni tra il nostro sistema e quelli esterni

\subsection{Utenti e sistemi esterni} % chi interagisce, chi usa e chi supporta (sia software che umano)
\subsubsection{User}
\subsubsection{Database}
\subsubsection{...}


\subsection{Diagramma di contesto} % frecce = dati che si scambiano. Utente --atuenticaz--> sistema (è l'utente che fornisce i dati al sistema; la freccia indica il verso del flusso)

% INVIO DI EMAIL per recupero password
% freccia da utente a sistema: richiesta recupero password
% freccia da sistema a utente: form
% freccia da utente a sistema: email nuova
% (di mezzo potrebbe essere meglio mettere il database che verifica l'email)
% freccia da sistema a servizio mail: INDIRIZZO MAIL untente, mittente, oggetto, testo, allegati (contenuto in generale) incluso LINK AL FORM DELLA PIATTAFORMA PER IL RESET PASSWORD

% 







\newpage
\section{Analisi dei componenti}

\end{document}