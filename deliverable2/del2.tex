\documentclass[11pt, a4paper]{article}

%\usepackage[T1]{fontenc}
%\usepackage{fullpage}

\usepackage[utf8]{inputenc} % comment when using lualatex
\usepackage[italian]{babel} % lingua e a-capo-sillabato
\usepackage{graphicx}
\usepackage[hidelinks]{hyperref} % link di pagina
\usepackage[bottom]{footmisc} % note appiccicate al fondo della pagina
\usepackage{float} % per posizionamento immagini
\usepackage{amsthm} % per ambienti stile teorema
\usepackage{tabularx} %tabelle
\usepackage[table]{xcolor} %colore caselle
\usepackage{enumitem} %additional commands for lists

\usepackage{fancyhdr}
\pagestyle{fancy}
\fancyhf{}% Clear header/footer
\fancyhead[C]{\footnotesize\textit{Documento:} D2 \hfill SleepCode \hfill \textit{Versione:} 1.0}
\renewcommand{\headrulewidth}{0.4pt}% Default \headrulewidth is 0.4pt


%\pagestyle{myheadings}
%\markright{John Smith\hfill On page styles\hfill}

\renewcommand\UrlFont{\color{blue}\rmfamily}

\theoremstyle{definition}

\newtheorem{funcreq}{RF} %% numerazione dei requisiti funzionali
\newtheorem{nonfuncreq}{RNF} %% requisiti non funzionali
\newtheorem{backend}{BE}
\newtheorem{frontend}{FE}

\title{Specifica dei Requisiti}

\author{Raffaele \textsc{Castagna}\\
Alberto \textsc{Rovesti}\\
Zeno \textsc{Saletti}}

\newcommand{\groupNumber}{G17}


% —

% Web address for the project (if any)
% \newcommand{\homepage}{\url{https://www.}}


% data
\date{\today}

\makeatletter{}

% IL PREAMBOLO FINISCE QUI %%%%%%%%%%%%%%%%%%%%%%%%%%%%%%%%%%%%%%%%%%%%%%%%%%%%






\begin{document}

% La pagina di copertina si trova in un file .tex a parte
% NON MODIFICARE QUESTO COMANDO!!!
\begin{titlepage}
\newcommand{\HRule}{\rule{\linewidth}{0.3mm}} % Defines a new command for horizontal lines, change thickness here
\center % Centre everything on the page

%------------------------------------------------
%	Logo
%------------------------------------------------
\includegraphics[width=0.3\textwidth]{materiale/UniTrento_logo_ITA_colore.png}\\[0.5cm]
%------------------------------------------------
%	Headings
%------------------------------------------------
\textsc{\Large Dipartimento di Ingegneria\\e Scienza dell'Informazione}\\[1.5cm]

{\Huge\textbf{Sleep Code}}\\[0.5cm]
\textsc{\large Progetto per il Corso di Ingegneria del Software}\\
\textsc{\large Anno Accademico 2023-2024}\\[0.5cm]

%------------------------------------------------
%	Title
%------------------------------------------------

\HRule\\[0.4cm]
{\huge\bfseries \@title}\\[0.1cm]
\HRule\\[1cm]

\begin{minipage}{\textwidth}
\begin{flushleft}
\textit{Descrizione:} documento di analisi dei requisiti funzionali, non funzionali, front-end e back-end.
\end{flushleft}
\end{minipage}\\[1.5cm]


\begin{minipage}{0.4\textwidth}
\begin{flushleft}
\large
\textit{Numero documento:} D1
\end{flushleft}
\end{minipage}
\begin{minipage}{0.4\textwidth}
\begin{flushright}
\large
\textit{Versione documento:} 2.4
\end{flushright}
\end{minipage}\\[1.5cm]

%------------------------------------------------
%	Author(s)
%------------------------------------------------
\begin{minipage}{0.4\textwidth}
\begin{flushleft}
\large
\textit{Membri del gruppo:}\\
\@author % Your name
\end{flushleft}
\end{minipage}
~
\begin{minipage}{0.4\textwidth}
\begin{flushright}
\large
\textit{Numero gruppo: }
\groupNumber
\end{flushright}
\end{minipage}

% 	If you don't want a supervisor, uncomment the two lines below and comment the code above
% 	{\large\textit{Author(s)}}\\
% 	\@author % Your name

%------------------------------------------------
%	Date
%------------------------------------------------

\vfill\vfill
\textit{Ultima revisione:}
{\@date}

\end{titlepage}

\tableofcontents

\newpage

\section*{Scopo del documento}
Il presente documento riporta la specifica dei requisiti di sistema
del progetto SleepCode ricorrendo a diagrammi realizzati secono lo
Unified Modeling Language (UML) e tabelle.

\section{Requisiti funzionali}
In questa sezione vengono descritti i requisiti funzionali (RF) del
servizio utilizzando alcuni Use Case Diagrams (UCD) scritti in UML,
eventualmente arricchiti da descrizioni in linguaggio naturale.

\section{Requisiti non funzionali}
Nella presente sezione sono riportati i requisiti non funzionali (RNF)
del sistema utilizzando tabelle strutturate e specificando misure che
consentano di effettuare facilmente delle verifiche quantitative.

\subsection{Caratteristiche di sistema}

\begin{nonfuncreq}
\textbf{Scalabilità }
\end{nonfuncreq}

\begin{nonfuncreq}
    \textbf{Compatibilità }
\end{nonfuncreq}

\begin{nonfuncreq}
    \textbf{Usabilità }
\end{nonfuncreq}

\begin{nonfuncreq}
    \textbf{Aspetto }
\end{nonfuncreq}

\begin{nonfuncreq}
    \textbf{Lingua }
    \begin{center}
        \footnotesize
        \begin{tabularx}{\textwidth}{|X||X||X|}
            \hline
            \cellcolor{red!70}Proprietà & \cellcolor{red!70}Descrizione & \cellcolor{red!70}Misura\\
            \hline
            Lingua di sistema           & Lingua presente nell'interfaccia e nelle risorse fornite dal servizio. & L'interfaccia generale della piattaforma conterrà testo in italiano (100\%); la totalità dei problemi può essere scritta scegliendo tra lingua italiana oppure inglese; le risorse multimediali (video-suggerimento) devono essere in italiano oppure in inglese.\\
            \hline
        \end{tabularx}
    \end{center}
\end{nonfuncreq}

\begin{nonfuncreq}
    \textbf{Disponibilità }
\end{nonfuncreq}

\begin{nonfuncreq}
    \textbf{Prestazioni }
\end{nonfuncreq}

\subsection{Privacy e sicurezza}

\begin{nonfuncreq}
    \textbf{Privacy e trattamento dei dati }
    \begin{center}
        \footnotesize
        \begin{tabularx}{\textwidth}{|X||X||X|}
            \hline
            \cellcolor{red!70}Proprietà & \cellcolor{red!70}Descrizione & \cellcolor{red!70}Misura\\
            \hline
            Normativa & Conformità con le vigenti norme relative al trattamento e alla protezione dei dati (GDPR). In particolare, i dati personali dell'utente registrato (nome, email e password) non devono essere divulgati in alcun modo e, qualora lo ritenga opportuno, l'utente ha il diritto di richiedere l'eliminazione delle proprie informazioni dal servizio al fine di interrompere il trattamento. & Conformità del servizio e funzionalità a supporto dell'utente (eliminazione account).\\
            \hline
        \end{tabularx}
    \end{center}
\end{nonfuncreq}

\begin{nonfuncreq}
    \textbf{Connessione sicura }
    \begin{center}
        \footnotesize
        \begin{tabularx}{\textwidth}{|X||X||X|}
            \hline
            \cellcolor{red!70}Proprietà & \cellcolor{red!70}Descrizione & \cellcolor{red!70}Misura\\
            \hline
            Connessione sicura & Impiego di protocolli di comunicazione che garantiscono la confidenzialità e riservatezza delle informazioni scambiate tra client e server. & Utilizzo del protocollo \texttt{https}.\\
            \hline
        \end{tabularx}
    \end{center}
\end{nonfuncreq}

\begin{nonfuncreq}
    \textbf{Password strength }
    \begin{center}
        \footnotesize
        \begin{tabularx}{\textwidth}{|X||X||X|}
            \hline
            \cellcolor{red!70}Proprietà & \cellcolor{red!70}Descrizione & \cellcolor{red!70}Misura\\
            \hline
            Password sicura & Quantità e varietà di caratteri necessari per comporre una password forte. & Una password conforme possiede almeno 8 caratteri (e non più di 64), tra i quali sono presenti almeno una lettera maiuscola, una minuscola, una cifra decimale e un carattere speciale.\\
            \hline
        \end{tabularx}
    \end{center}
\end{nonfuncreq}

\section{Analisi del contesto}
\section{Analisi dei componenti}

\end{document}