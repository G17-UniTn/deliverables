\documentclass[11pt, a4paper]{article}

%\usepackage[T1]{fontenc}
%\usepackage{fullpage}

\usepackage[utf8]{inputenc} % comment when using lualatex
\usepackage[italian]{babel} % lingua e a-capo-sillabato
\usepackage{graphicx}
\usepackage[hidelinks]{hyperref} % link di pagina
\usepackage[bottom]{footmisc} % note appiccicate al fondo della pagina
\usepackage{float} % per posizionamento immagini
\usepackage{amsthm} % per ambienti stile teorema
\usepackage{tabularx} %tabelle
\usepackage[table]{xcolor} %colore caselle
\usepackage{enumitem} %additional commands for lists
\usepackage{fancyhdr}
\usepackage[font=footnotesize,labelfont=bf]{caption} % small caption font-size



\pagestyle{fancy}
\fancyhf{}% Clear header/footer
\fancyhead[C]{\footnotesize\textit{Documento:} D3 \hfill SleepCode \hfill \textit{Versione:} 1.0}
\renewcommand{\headrule}{{\color{red!70}\rule{\textwidth}{2pt}}}
\setlength{\headheight}{22pt}

\renewcommand\UrlFont{\color{blue}\rmfamily} % colore link

\theoremstyle{definition} % stile dei newtheorem (non italizzati)
\newtheorem{funcreq}{RF} %% numerazione dei requisiti funzionali
\newtheorem{nonfuncreq}{RNF} %% requisiti non funzionali
\newtheorem{backend}{BE}
\newtheorem{frontend}{FE}




\title{Documento di Architettura}

\author{Raffaele \textsc{Castagna}\\
Alberto \textsc{Rovesti}\\
Zeno \textsc{Saletti}}

\newcommand{\groupNumber}{G17}

% Web address for the project (if any)
% \newcommand{\homepage}{\url{https://www.}}

% data
\date{\today}

\makeatletter{}

% IL PREAMBOLO FINISCE QUI %%%%%%%%%%%%%%%%%%%%%%%%%%%%%%%%%%%%%%%%%%%%%%%%%%%%



\begin{document}

% La pagina di copertina si trova in un file .tex a parte
% NON MODIFICARE QUESTO COMANDO!!!
\begin{titlepage}
\newcommand{\HRule}{\rule{\linewidth}{0.3mm}} % Defines a new command for horizontal lines, change thickness here
\center % Centre everything on the page

%------------------------------------------------
%	Logo
%------------------------------------------------
\includegraphics[width=0.3\textwidth]{materiale/UniTrento_logo_ITA_colore.png}\\[0.5cm]
%------------------------------------------------
%	Headings
%------------------------------------------------
\textsc{\Large Dipartimento di Ingegneria\\e Scienza dell'Informazione}\\[1.5cm]

{\Huge\textbf{Sleep Code}}\\[0.5cm]
\textsc{\large Progetto per il Corso di Ingegneria del Software}\\
\textsc{\large Anno Accademico 2023-2024}\\[0.5cm]

%------------------------------------------------
%	Title
%------------------------------------------------

\HRule\\[0.4cm]
{\huge\bfseries \@title}\\[0.1cm]
\HRule\\[1cm]

\begin{minipage}{\textwidth}
\begin{flushleft}
\textit{Descrizione:} documento di analisi dei requisiti funzionali, non funzionali, front-end e back-end.
\end{flushleft}
\end{minipage}\\[1.5cm]


\begin{minipage}{0.4\textwidth}
\begin{flushleft}
\large
\textit{Numero documento:} D1
\end{flushleft}
\end{minipage}
\begin{minipage}{0.4\textwidth}
\begin{flushright}
\large
\textit{Versione documento:} 2.4
\end{flushright}
\end{minipage}\\[1.5cm]

%------------------------------------------------
%	Author(s)
%------------------------------------------------
\begin{minipage}{0.4\textwidth}
\begin{flushleft}
\large
\textit{Membri del gruppo:}\\
\@author % Your name
\end{flushleft}
\end{minipage}
~
\begin{minipage}{0.4\textwidth}
\begin{flushright}
\large
\textit{Numero gruppo: }
\groupNumber
\end{flushright}
\end{minipage}

% 	If you don't want a supervisor, uncomment the two lines below and comment the code above
% 	{\large\textit{Author(s)}}\\
% 	\@author % Your name

%------------------------------------------------
%	Date
%------------------------------------------------

\vfill\vfill
\textit{Ultima revisione:}
{\@date}

\end{titlepage}

\tableofcontents

\newpage

\section*{Scopo del documento}
In questo documento viene riportata la definizione dell'architettura del
progetto \textit{SleepCode} impiegando diagrammi delle classi, realizzati
secondo gli standard di Unified Modeling Language (UML), e codice Object
Constraint Language (OCL). Nel documento precedente (D2, \textit{Specifica
dei Requisiti}) sono stati presentati il diagramma degli use case, quello
di contesto e infine il diagramma dei componenti. Considerando tale
progettazione, viene ora definita l'architettura del sistema specificando
in modo più dettagliato le classi che dovranno essere implementate sotto forma di
codice, insieme alla logica che regola il comportamento del software che
si intende realizzare.

Il linguaggio UML, utilizzato per descrivere le classi, è supportato da
codice OCL, impiegato invece per specificare gli aspetti logici citati sopra
che non sarebbero esprimibili formalmente mediante i soli diagrammi delle
classi.


\newpage
\section{Diagramma delle classi}
Nella presente sezione vengono illustrate in linguaggio UML le classi
previste dal progetto \textit{SleepCode}. Ogni componente del diagramma
dei componenti, presente nel documento D2, viene qui rappresentato
in forma di una o più classi. Le classi individuate sono costituite
da un nome, un insieme di attributi che identificano i dati gestiti dalla
classe stessa e una lista di metodi che definiscono le operazioni
eseguibili da quella classe. Eventuali relazioni tra classi sono evidenziate
da alcune associazioni.


\section{Codice in Object Constraint Language}
Nella sezione che segue viene descritta formalmente la logica prevista
nel comportamento di alcune classi, in relazione alle loro operazioni
possibili. Il codice OCL impiegato consente di esprimere tale logica,
non descrivibile con i soli diagrammi delle classi in UML.


\section{Diagramma delle classi con codice OCL}
In Figura \ref{umlocl} è riportato un diagramma che mostra nell'insieme
sia le classi individuate che il relativo codice OCL.

\begin{figure}[H]
\centering
%\includegraphics[options]{name}
\caption{Diagramma delle classi arricchito dal codice OCL}
\label{umlocl}
\end{figure}

\end{document}