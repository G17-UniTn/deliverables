\documentclass[11pt, a4paper]{article}

%\usepackage[T1]{fontenc}
\usepackage[utf8]{inputenc} % comment when using lualatex
\usepackage[italian]{babel}
\usepackage{fullpage}
\usepackage{graphicx}
\usepackage[hidelinks]{hyperref,xcolor} % link di pagina
\usepackage[bottom]{footmisc} % note appiccicate al fondo della pagina

\renewcommand\UrlFont{\color{blue}\rmfamily}

\usepackage{amsthm}
\theoremstyle{definition}

\newtheorem{funcreq}{RF} %% numerazione dei requisiti funzionali
\newtheorem{nonfuncreq}{RNF} %% requisiti non funzionali


\title{Analisi dei Requisiti}

\author{Raffaele \textsc{Castagna}\\
Alberto \textsc{Rovesti}\\
Zeno \textsc{Saletti}}

\newcommand{\groupNumber}{G17}

% Web address for the project (if any)
% \newcommand{\homepage}{\url{https://www.}}

% data
\date{\today}

\makeatletter{}

% IL PREAMBOLO FINISCE QUI %%%%%%%%%%%%%%%%%%%%%%%%%%%%%%%%%%%%%%%%%%%%%%%%%%%%






\begin{document}

% La pagina di copertina si trova in un file .tex a parte
\begin{titlepage}
\newcommand{\HRule}{\rule{\linewidth}{0.3mm}} % Defines a new command for horizontal lines, change thickness here
\center % Centre everything on the page

%------------------------------------------------
%	Logo
%------------------------------------------------
\includegraphics[width=0.3\textwidth]{materiale/UniTrento_logo_ITA_colore.png}\\[0.5cm]
%------------------------------------------------
%	Headings
%------------------------------------------------
\textsc{\Large Dipartimento di Ingegneria\\e Scienza dell'Informazione}\\[1.5cm]

{\Huge\textbf{Sleep Code}}\\[0.5cm]
\textsc{\large Progetto per il Corso di Ingegneria del Software}\\
\textsc{\large Anno Accademico 2023-2024}\\[0.5cm]

%------------------------------------------------
%	Title
%------------------------------------------------

\HRule\\[0.4cm]
{\huge\bfseries \@title}\\[0.1cm]
\HRule\\[1cm]

\begin{minipage}{\textwidth}
\begin{flushleft}
\textit{Descrizione:} documento di analisi dei requisiti funzionali, non funzionali, front-end e back-end.
\end{flushleft}
\end{minipage}\\[1.5cm]


\begin{minipage}{0.4\textwidth}
\begin{flushleft}
\large
\textit{Numero documento:} D1
\end{flushleft}
\end{minipage}
\begin{minipage}{0.4\textwidth}
\begin{flushright}
\large
\textit{Versione documento:} 2.4
\end{flushright}
\end{minipage}\\[1.5cm]

%------------------------------------------------
%	Author(s)
%------------------------------------------------
\begin{minipage}{0.4\textwidth}
\begin{flushleft}
\large
\textit{Membri del gruppo:}\\
\@author % Your name
\end{flushleft}
\end{minipage}
~
\begin{minipage}{0.4\textwidth}
\begin{flushright}
\large
\textit{Numero gruppo: }
\groupNumber
\end{flushright}
\end{minipage}

% 	If you don't want a supervisor, uncomment the two lines below and comment the code above
% 	{\large\textit{Author(s)}}\\
% 	\@author % Your name

%------------------------------------------------
%	Date
%------------------------------------------------

\vfill\vfill
\textit{Ultima revisione:}
{\@date}

\end{titlepage}

\tableofcontents


\newpage
\section{Introduzione e scopi}
\subsection{Scopo del documento}
Le informazioni contenute in questo documento concorrono ad esporre l'analisi
dei requisiti relativa al progetto \textit{SleepCode}. In particolare, dopo
aver specificato gli obiettivi e gli attori coinvolti—utenti finali e
utilizzatori del frutto di questo progetto—verranno definiti i requisiti
funzionali e non funzionali; verrà presentata una proposta di design di
back-end; infine saranno riportati i sistemi di back-end coi quali il servizio
dovrà interfacciarsi.


\subsection{Obiettivo del progetto}
Il progetto proposto si prefigge, come scopo fondante, di fornire alla comunità
di giovani informatici un servizio online di \textit{esercitazione} mirata alla
programmazione e alla progettazione di piccoli algoritmi risolutivi, mediante
la scrittura di codice.

\subsection{Attori coinvolti ed esigenze}
Per comprendere meglio i requisiti che verranno descritti in seguito (in
particolar modo quelli funzionali), è innanzitutto essenziale specificare
il pubblico, insieme alle loro potenziali esigenze, al quale il servizio
intende rivolgersi.
Tale servizio vuole rendersi utile soprattutto a coloro che sono coinvolti
in percorsi di studio attinenti all'ambito informatico, ma specialmente anche
a chiunque desideri cimentarsi nella risoluzione di piccoli problemi di
programmazione; pertanto ci si aspetta che chiunque desideri usufruire del
servizio possieda almeno le conoscenze basilari della programmazione. Esempi
di queste nozioni pregresse, che tuttavia non devono necessariamente essere ampie e approfondite per utilizzare il servizio\footnote{Gli utenti più esperti possono indubbiamente trarre vantaggio dal loro bagaglio culturale per
approcciarsi con maggior facilità al servizio.}, sono: cosa si intende per
algoritmo e linguaggio di programmazione, familiarità nell'uso di qualche
linguaggio di programmazione, tipi e strutture di dati più comuni.

Di fatto, il progetto che verrà sviluppato ha come scopo principale di
creare una piattaforma accessibile online a singoli utilizzatori che
desiderano esercitarsi, valutare e approfondire le personali
conoscenze e abilità di \textit{problem solving} legate alla programmazione.
D'ora in avanti, in questo e nei successivi documenti, questo pubblico di
individui appena descritti verrà indicato con il termine \textit{utenti}.

\newpage
\section{Requisiti funzionali}
Vengono di seguito elencati i principali requisiti funzionali (RF)
del progetto. Ogni sottosezione di questa parte del documento
risponde a diversi scopi precedentemente accennati, suddividendo
eventuali macro-funzioni in requisiti minori nel caso di obiettivi
di più ampia portata.

\subsection{Accesso}

\begin{funcreq}
\label{signup}
\textbf{Registrazione }
Il sistema deve mettere a disposizione dei nuovi utenti una pagina
nella quale sia possibile effettuare la registrazione alla piattaforma
online. Devono essere richiesti i seguenti campi:
\begin{itemize}
    \item Indirizzo email.
    \item Nome utente.
    \item Password: la password deve essere oscurata durante l'inserimento.
    Al fine di limitare gli errori di digitazione e l'inserimento di una
    password che non corrisponde a quella scelta dall'utente, deve essere
    presente un secondo campo di conferma, dove la password deve essere
    riscritta. La password deve in ogni caso essere conforme a quanto
    specificato nel RNF \ref{legalpassword}.
\end{itemize}
\end{funcreq}

\begin{funcreq}
\label{login}
\textbf{Login }
Il sistema deve permettere all'utente già registrato al servizio di
accedere alla piattaforma mediante l'inserimento dell'indirizzo
email e la password, impostate in fase di registrazione. Qualora la
password inserita non corrisponda a quella dell'account con email
associata, l'utente non può accedere alla piattaforma e deve essere
notificato di ciò; è comunque possibile ritentare il login un numero illimitato di volte.
\end{funcreq}

\begin{funcreq}
\label{savepassword}
\textbf{Recupero password }
L'utente registrato deve poter recuperare la password del proprio account
qualora tale dato dovesse essere dimenticato in fase di login. La procedura
di recupero deve prevedere:
\begin{itemize}
    \item L'invio di una mail contenente link apposito dove poter modificare la password corrente
    ad un'altra password conforme al RNF \ref*{legalpassword}.

    \item L'inserimento di una nuova password che, come in fase di registrazione,
    deve essere scritta e confermata.
\end{itemize}
\end{funcreq}

\begin{funcreq}
\label{logout}
\textbf{Logout }
L'utente registrato e autenticato deve poter interrompere la sessione di
accesso al servizio effettuando un logout.
\end{funcreq}

\subsection{Consultazione dei problemi}

\begin{funcreq}
\label{probcatalogue}
\textbf{Consultazione del catalogo dei problemi }
Il servizio deve mettere a disposizione un insieme di problemi sui quali
l'utente possa esercitarsi. L'utente deve poter consultare un catalogo,
atto a raccogliere i quesiti, e navigare al suo interno. Deve quindi
essere possibile:
\begin{enumerate}
    \item Visualizzare tale catalogo in una vista dedicata.
    
    \item Cercare uno o più problemi specifici mediante ricerca filtrata
    per campi, specificati dai metadati elencati al RNF \ref{metadata},
    oppure non filtrata.

    \item Selezionare dal catalogo un problema specifico, cosicché la
    descrizione dell'intero problema possa essere visionata per mezzo
    delle funzionalità di cui al RF \ref{seeproblem}.
\end{enumerate}
\end{funcreq}

\begin{funcreq}
\label{seeproblem}
\textbf{Consultazione di un problema }
Deve essere fornito un visualizzatore del documento contenente le
informazioni del singolo problema precedentemente selezionato dal
catalogo. La visualizzazione deve rendere disponibile alla vista
dell'utente i dati relativi al problema e specificati nel RNF
\ref{formatoproblema}.
\end{funcreq}

\subsection{Esercitazione}
\begin{funcreq}
\label{exesession}
\textbf{Avviare una sessione di esercitazione }
L'utente deve poter selezionare, attraverso l'apposito catalogo, il
problema desiderato, dopodichè dovra essere portato ad una pagina apposita, dove
l'utente potrà vedere il testo dell'esercizio, i test case per la verifica della correttezza dell'arlgoritmo
come descritto in RF \ref*{sintax}, una console dove scrivere il codice per risolvere il problema selezionato, e testarlo,
e appositi selettori per il linguaggio di programmazione desiderato. A tal fine, l'utente deve poter:
\begin{enumerate}
    \item Attivare e fermare a proprio piacimento un timer apposito per cronometrare il tempo speso
    per risolvere il quesito scelto.
    
    \item Essere al corrente di quale linguaggio di programmazione sia
    attualmente attivo per la scrittura di codice, tramite un menu dedicato
    dal quale deve altresì essere possibile selezionare uno dei linguaggi
    disponibili\footnote{Per approfondimenti sulla disponibilità dei linguaggi, si veda RNF \ref{scalabilita}.1.}.
     //(Non dovrebbe far parte dei RNF? Stiamo parlando di UI design, la parte del "menu" sarebbe da incorporare nel testo principale imo)

    
    \item Scrivere, sotto forma di codice nel linguaggio di programmazione
    scelto, l'algoritmo risolutivo del problema attualmente attivo. La
    scrittura deve poter essere effettuata in una vista o finestra apposita.
\end{enumerate}
\end{funcreq}

\begin{funcreq}
\label{sintax}
\textbf{Correttezza sintattica del codice }
L'utente deve poter verificare che il codice scritto sia corretto e in grando di risolvere il problema.
 Quindi devono essere messe a disposizione le seguenti funzionalità:
\begin{enumerate}
    \item Compilazione del codice.
    \item Visualizzazione di avvisi relativi a eventuali errori di compilazione
    \textit{oppure} di compilazione andata a buon fine. In caso di errori
    di scrittura, l'utente deve poter correggere tali errori riscrivendo
    nell'area destinata al codice.
\end{enumerate}
\end{funcreq}

\begin{funcreq}
\label{test}
\textbf{Verifica della correttezza dell'algoritmo\footnote{La \textit{correttezza} di cui si parla in questo caso riguarda solo l'efficacia risolutiva dell'algoritmo.} }
L'utente deve poter verificare la correttezza del codice scritto eseguendolo
e testandolo:
\begin{enumerate}
    \item Il codice deve essere eseguito sottoponendolo ad un certo numero
    di test cases (al minimo 3),
    cioè fornendo opportune istanze di input (Per esempio,
    se un problema richiede di sommare due numeri interi, il codice risolutivo
    proposto dall'utente verrà eseguito fornendo ad esso coppie di interi
    e registrando i risultati in output.)

    \item L'utente deve essere in grado di essere notificato se il codice supera tutti i test case oppure se almeno 1
    test case non è stato superato,in entrambi i casi l'utente verrà notificato attraverso un pop-up apposito. Per ogni test case, sia l'input che l'output desiderato sono visibili.
    Inoltre, l'utente deve poter riscrivere e
    perfezionare l'algoritmo e sottoporre ripetutamente il codice ai
    test cases.

    \item Nel caso che l'utente autenticato sia in grado di superare tutti i test case disponibili, il problema dovrà
    essere contrassegnato come "risolto" nel catalogo di problemi.
\end{enumerate}
\end{funcreq}

\subsection{Gestione profilo e account}

\begin{funcreq}
\label{stats}
\textbf{Progressi }
L'utente deve poter monitorare i propri progressi, accedendo ai dati
seguenti: 
\begin{itemize}
    \item Problemi risolti suddivisi per difficoltà
\end{itemize}
\end{funcreq}

\begin{funcreq}
\textbf{Preferiti}
    \begin{itemize}
        \item L'utente deve essere in grado di poter aggiungere ad una lista di "preferiti"
        qualsiasi problema, attraverso un apposito bottone.
    \end{itemize}
\end{funcreq}


\begin{funcreq}
\label{updateaccount}
\textbf{Aggiornamento account }
L'utente registrato e autenticato deve poter modificare i dati chiave del
proprio account in conformità con i requisiti non funzionali relativi alla
sicurezza [...]:
\begin{enumerate}
\item L'utente deve poter migrare ad un indirizzo email differente. A tale
indirizzo deve essere inviato un messaggio di conferma, previo inserimento
della password nella pagina dedicata a questa particolare operazione. L'email
deve cambiare effettivamente solo nel momento in cui l'utente conferma la
ricezione del messaggio presso il nuovo indirizzo.

\item L'utente deve poter modificare la password del proprio account.
La password può essere cambiata previo inserimento di quella attualmente
associata all'account. La nuova password deve essere digitata due volte
per conferma e deve rispettare quanto specificato dal RNF \ref{legalpassword}. 
L'utente deve essere notificato di tale modifica anche tramite
l'indirizzo email attualmente associato all'account [possibilità di
disattivare account a causa di azioni sospette?].
\end{enumerate}
\end{funcreq}


\newpage
\section{Requisiti non funzionali}
Vengono ora elencati i requisiti non funzionali (RNF) del servizio.

\subsection{Caratteristiche delle risorse}
Alcuni requisiti funzionali sono dedicati alla descrizione delle principali
risorse fornite dalla piattaforma, di cui le principali sono i problemi.

\begin{nonfuncreq}
\label{formatoproblema}
\textbf{Struttura di un problema }
Tutti i problemi forniti possiedono:
\begin{itemize}
    \item Un titolo.

    \item Un testo, scritto prevalentemente in linguaggio naturale,
    che descrive uno scenario che richiede di essere risolto per mezzo
    di un algoritmo. Eventuali immagini e proposizioni matematiche
    formali possono accompagnare i testi.

    \item Almeno 3 esempi di input insieme al relativo output che
    mostra il risultato atteso.
\end{itemize}
\end{nonfuncreq}

\begin{nonfuncreq}
\label{metadata}
\textbf{Metadati dei problemi } Ogni problema è essenzialmente un'entità
caratterizzata dai seguenti dati descrittivi:
\begin{itemize}
    \item Stato: i problemi già risolti dall'utente sono contrassegnati
    in un campo \textit{stato} apposito.

    \item Nome identificativo: si tratta di una stringa alfanumerica, che
    riprende una o più parole chiave del titolo del problema di cui al
    RNF \ref{formatoproblema}.
    
    \item Difficoltà: ogni problema possiede un'etichetta (\textit{tag})
    associata che ne valuta indicativamente la difficolta: bassa,
    intermedia e alta.

    \item Categoria: la categoria indica la principale area di interesse
    del problema, come ad esempio le strutture di dati di cui tratta
    (array, grafi, code, ecc.).

    \item Soluzione: un video apposito, non necessariamente in italiano dove poter vedere 
    come il problema doveva essere svolto.
\end{itemize}
\end{nonfuncreq}

\subsection{Caratteristiche di sistema}

\begin{nonfuncreq}
\label{scalabilita}
\textbf{Scalabilità }
L'infrastruttura del servizio deve essere scalabile e aperta alle esigenze
derivanti dall'aumento di nuovi utenti. Questo requisito è motivato dalla
disponibilità online del servizio che verrà sviluppato. In particolare:
\begin{enumerate}
    \item L'infrastruttura del servizio deve essere adattabile a eventuali
    crescite nel numero di utenti, in modo da prevenire possibili cali di
    prestazioni eccessivi.

    \item Data l'eterogeneità di linguaggi di programmazione esistenti
    al momento della stesura di questo documento, è importante che il
    servizio sia in grado di accogliere con l'avanzare del tempo codici
    scritti in linguaggi differenti.
\end{enumerate}
\end{nonfuncreq}

\begin{nonfuncreq}
\label{compatibility}
\textbf{Compatibilità }
La piattaforma del servizio deve essere accessibile mediante i principali
browser attualmente disponibili in commercio.
\end{nonfuncreq}


\begin{nonfuncreq}
\textbf{Usabilità }
La piattaforma del servizio deve permettere all'utente di sfruttare le
funzionalità disponibili al proprio livello di accesso senza l'ausilio di
istruzioni scritte e verbose. L'intuitività dell'interfaccia deve essere
sufficiente a guidare l'utente nella realizzazione dei suoi scopi.
\end{nonfuncreq}

\begin{nonfuncreq}
\textbf{Aspetto }
L'interfaccia deve presentarsi gradevole alla vista dell'utente, preferendo
gradazioni cromatiche scure e un contrasto sufficientemente equilibrato,
al fine di garantire la leggibilità e contribuire alla riduzione
dell'affaticamento della vista.
\end{nonfuncreq}

\begin{nonfuncreq}
\textbf{Lingua di sistema }
Il servizio sarà erogato in lingua italiana. Altrettanto sarà fatto per i
testi dei problemi.
\end{nonfuncreq}

\begin{nonfuncreq}
\textbf{Affidabilità }

\end{nonfuncreq}

\begin{nonfuncreq}
\textbf{Prestazioni }

\end{nonfuncreq}

\subsection{Privacy e sicurezza}

\begin{nonfuncreq}
\textbf{Privacy e trattamento dei dati }
Il servizio deve essere progettato e realizzato in ottemperanza delle
vigenti disposizioni di legge in materia di tutela della privacy e
trattamento dei dati:
\begin{enumerate}
    \item L'applicazione fornita dal servizio deve essere conforme
    al regolamento \href{https://www.garanteprivacy.it/documents/10160/0/Regolamento+UE+2016+679.+Arricchito+con+riferimenti+ai+Considerando+Aggiornato+alle+rettifiche+pubblicate+sulla+Gazzetta+Ufficiale++dell%27Unione+europea+127+del+23+maggio+2018}{\textcolor{blue}{\underbar{UE n.2016/679}}} per la protezione dei dati.
\end{enumerate}
\end{nonfuncreq}

\begin{nonfuncreq}
\label{legalpassword}
\textbf{Password }
In tutti gli scenari nei quali è richiesto l'inserimento di una password,
devono essere rispettate le seguenti caratteristiche:
\begin{itemize}
    \item Lunghezza compresa tra 8 e 64 caratteri.
    \item Contenere almeno una lettera maiuscola.
    \item Contenere almeno una lettera minuscola.
    \item Contenere almeno un numero.
    \item Contenere almeno un carattere speciale scelto tra i
    seguenti:
    \begin{center}
        \verb|! ? # $ % & @ * + . , ; : / - = _ \ ( ) [ ] { }|
    \end{center}
\end{itemize}
\end{nonfuncreq}

\begin{nonfuncreq}
\textbf{Integrità e operazioni su dati sensibili }
L'integrità di dati sensibili dell'account, quali email ma soprattutto password,
è garantita da opportune procedure di conferma e notifica all'utente.
Data la natura di queste operazioni di sicurezza, i requisiti funzionali
relativi alla gestione dei dati sensibili (RF \ref{signup}, RF \ref{savepassword},
RF \ref{updateaccount})
provvedono a descrivere il comportamento del servizio nel caso di errori
e incongruenze.
\end{nonfuncreq}


\newpage
\section{Relazioni tra obiettivi e requisiti}
[tabella che associa obiettivi e requisiti?]
\section{Design front-end}
\section{Design back-end}

\end{document}
