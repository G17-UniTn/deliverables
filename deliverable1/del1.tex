\documentclass[11pt, a4paper]{article}

%\usepackage[T1]{fontenc}
%\usepackage{fullpage}

\usepackage[utf8]{inputenc} % comment when using lualatex
\usepackage[italian]{babel} % lingua e a-capo-sillabato
\usepackage{graphicx}
\usepackage[hidelinks]{hyperref,xcolor} % link di pagina
\usepackage[bottom]{footmisc} % note appiccicate al fondo della pagina
\usepackage{float} % per posizionamento immagini
\usepackage{amsthm}

\usepackage{fancyhdr}
\pagestyle{fancy}
\fancyhf{}% Clear header/footer
\fancyhead[C]{\footnotesize\textit{Documento:} D1 \hfill SleepCode \hfill \textit{Versione:} 2.1}
\renewcommand{\headrulewidth}{0.4pt}% Default \headrulewidth is 0.4pt


%\pagestyle{myheadings}
%\markright{John Smith\hfill On page styles\hfill}

\renewcommand\UrlFont{\color{blue}\rmfamily}

\theoremstyle{definition}

\newtheorem{funcreq}{RF} %% numerazione dei requisiti funzionali
\newtheorem{nonfuncreq}{RNF} %% requisiti non funzionali
\newtheorem{backend}{BE}
\newtheorem{frontend}{FE}

\title{Analisi dei Requisiti}

\author{Raffaele \textsc{Castagna}\\
Alberto \textsc{Rovesti}\\
Zeno \textsc{Saletti}}

\newcommand{\groupNumber}{G17}


% —

% Web address for the project (if any)
% \newcommand{\homepage}{\url{https://www.}}

% data
\date{\today}

\makeatletter{}

% IL PREAMBOLO FINISCE QUI %%%%%%%%%%%%%%%%%%%%%%%%%%%%%%%%%%%%%%%%%%%%%%%%%%%%






\begin{document}

% La pagina di copertina si trova in un file .tex a parte
% NON MODIFICARE QUESTO COMANDO!!!
\begin{titlepage}
\newcommand{\HRule}{\rule{\linewidth}{0.3mm}} % Defines a new command for horizontal lines, change thickness here
\center % Centre everything on the page

%------------------------------------------------
%	Logo
%------------------------------------------------
\includegraphics[width=0.3\textwidth]{materiale/UniTrento_logo_ITA_colore.png}\\[0.5cm]
%------------------------------------------------
%	Headings
%------------------------------------------------
\textsc{\Large Dipartimento di Ingegneria\\e Scienza dell'Informazione}\\[1.5cm]

{\Huge\textbf{Sleep Code}}\\[0.5cm]
\textsc{\large Progetto per il Corso di Ingegneria del Software}\\
\textsc{\large Anno Accademico 2023-2024}\\[0.5cm]

%------------------------------------------------
%	Title
%------------------------------------------------

\HRule\\[0.4cm]
{\huge\bfseries \@title}\\[0.1cm]
\HRule\\[1cm]

\begin{minipage}{\textwidth}
\begin{flushleft}
\textit{Descrizione:} documento di analisi dei requisiti funzionali, non funzionali, front-end e back-end.
\end{flushleft}
\end{minipage}\\[1.5cm]


\begin{minipage}{0.4\textwidth}
\begin{flushleft}
\large
\textit{Numero documento:} D1
\end{flushleft}
\end{minipage}
\begin{minipage}{0.4\textwidth}
\begin{flushright}
\large
\textit{Versione documento:} 2.4
\end{flushright}
\end{minipage}\\[1.5cm]

%------------------------------------------------
%	Author(s)
%------------------------------------------------
\begin{minipage}{0.4\textwidth}
\begin{flushleft}
\large
\textit{Membri del gruppo:}\\
\@author % Your name
\end{flushleft}
\end{minipage}
~
\begin{minipage}{0.4\textwidth}
\begin{flushright}
\large
\textit{Numero gruppo: }
\groupNumber
\end{flushright}
\end{minipage}

% 	If you don't want a supervisor, uncomment the two lines below and comment the code above
% 	{\large\textit{Author(s)}}\\
% 	\@author % Your name

%------------------------------------------------
%	Date
%------------------------------------------------

\vfill\vfill
\textit{Ultima revisione:}
{\@date}

\end{titlepage}

\tableofcontents


\newpage
\section{Introduzione e scopi}
\subsection{Scopo del documento}
Le informazioni contenute in questo documento concorrono ad esporre l'analisi
dei requisiti relativa al progetto \textit{SleepCode}. In particolare, dopo
aver specificato gli obiettivi e gli attori coinvolti, verranno definiti i requisiti
funzionali e non funzionali; verrà presentata una proposta di design di
back-end; infine saranno riportati i sistemi esterni di back-end coi quali il servizio
dovrà interfacciarsi.


\subsection{Obiettivo del progetto}
Il progetto proposto si prefigge, come scopo fondante, di fornire alla comunità
di giovani informatici un servizio online di:
\begin{itemize}
    \item \textit{esercitazione} mirata alla programmazione e alla progettazione
    di piccoli algoritmi risolutivi, mediante la scrittura di codice;
    \item \textit{raccolta} di problemi attinenti.
\end{itemize}

\subsection{Pubblico di riferimento ed esigenze}
Il servizio vuole rendersi utile soprattutto a coloro che sono coinvolti
in percorsi di studio attinenti all'ambito informatico, ma specialmente anche
a chiunque desideri cimentarsi nella risoluzione di piccoli problemi di
programmazione; pertanto ci si aspetta che chiunque desideri usufruire del
servizio possieda almeno le conoscenze basilari della programmazione. Esempi
di queste nozioni pregresse, che tuttavia non devono necessariamente essere ampie e approfondite per utilizzare il servizio\footnote{Gli utenti più esperti possono indubbiamente trarre vantaggio dal loro bagaglio culturale per
approcciarsi con maggior facilità al servizio.}, sono: cosa si intende per
algoritmo e linguaggio di programmazione, familiarità nell'uso di qualche
linguaggio di programmazione, tipi e strutture di dati più comuni.

Di fatto, il progetto che verrà sviluppato ha come scopo principale di
creare una piattaforma accessibile online a singoli utilizzatori che
desiderano esercitarsi, valutare e approfondire le personali
conoscenze e abilità di \textit{problem solving} legate alla programmazione,
insieme all'archiviazione di nuovi problemi da rendere disponibili a coloro
che intendono usufruire del servizio.
D'ora in avanti, in questo e nei successivi documenti, questo pubblico di
individui appena descritti verrà indicato con il termine \textit{utenti}.

\newpage
\section{Requisiti funzionali}
Vengono di seguito elencati i principali requisiti funzionali (RF)
del progetto. Ogni sottosezione di questa parte del documento
risponde a diversi scopi precedentemente accennati, suddividendo
eventuali macro-funzioni in requisiti minori nel caso di obiettivi
di più ampia portata.
\\\\
\noindent Le funzionalità sono inoltre suddivise secondo alcuni livelli di accesso
degli utenti:
\begin{itemize}
    \item \textit{Utente anonimo.}
    \item \textit{Utente autenticato.}
    \item \textit{Utente amministratore.}
\end{itemize}
Si precisa che le funzionalità utilizzabili dall'utente anonimo sono condivise
con l'utente autenticato. L'utente amministratore non può accedere ad alcuna
funzionalità descritta all'infuori della propria sezione; gli utenti anonimo e
autenticato non possono usufruire delle funzionalità presenti a livello amministratore.

\begin{center}
\section*{Utente anonimo}    
\end{center}


\subsection{Accesso}

\begin{funcreq}
\label{signup}
\textbf{Registrazione }
I nuovi utenti devono poter registrarsi al servizio, ovvero creare un account che
richiede la compilazione dei seguenti campi:
\begin{itemize}
    \item Indirizzo email.
    \item Nome utente.
    \item Password: data la sensibilità di questo campo, è importante
    che siano rispettati i vincoli del \textcolor{blue}{\underbar{\hyperref[legalpassword]{RNF \ref*{legalpassword}}}};
    inoltre, l'utente deve poter riscrivere la password per confermarla, in
    modo tale da rilevare eventuali errori di digitazione.
\end{itemize}
L'utente dotato di un account sarà considerato dal servizio come
\textit{utente registrato} (indipendentemente dal suo livello di accesso).
\end{funcreq}

\begin{funcreq}
\label{login}
\textbf{Login }
Il sistema deve permettere all'utente già registrato al servizio di
accedere alla piattaforma mediante l'inserimento dell'indirizzo
email e la password, impostate in fase di registrazione. L'utente deve
essere notificato qualora non sia possibile concludere correttamente
l'operazione di login, quindi se l'email inserita non sia registrata o
la password non corrisponda; è comunque possibile ritentare il login un
numero illimitato di volte.

Effettuato il login, l'utente registrato e anonimo rientra nella categoria degli
\textit{utenti autenticati}.
\end{funcreq}

\begin{funcreq}
\label{savepassword}
\textbf{Recupero password }
L'utente registrato deve poter recuperare la password del proprio account
qualora tale dato dovesse essere dimenticato in fase di login. La procedura
di recupero deve prevedere:
\begin{itemize}
    \item L'invio di un messaggio all'indirizzo email attualmente associato
    all'account dell'utente; l'email contiene un link che guida l'utente
    nella creazione di una nuova password.

    \item L'inserimento (sulla piattaforma) di una nuova password che,
    come in fase di registrazione, deve essere scritta, confermata ed essere
    conforme al \textcolor{blue}{\underbar{\hyperref[legalpassword]{RNF \ref*{legalpassword}}}}.
\end{itemize}
\end{funcreq}

\subsection{Consultazione dei problemi}

\begin{funcreq}
\label{probcatalogue}
\textbf{Consultazione del catalogo dei problemi }
Il servizio deve mettere a disposizione un insieme di problemi sui quali
l'utente possa esercitarsi. L'utente deve poter consultare un catalogo,
atto a raccogliere i quesiti, e navigare al suo interno. Deve quindi
essere possibile:
\begin{enumerate}
    \item Visualizzare tale catalogo, che mostra i campi di ogni problema
    elencato:
    \begin{itemize}
        \item \textit{Stato:} l'utente anonimo non può visualizzare
        questo campo. Per l'utente autenticato, lo stato del problema
        viene marcato come \textit{risolto} qualora il problema sia
        stato
        \item \textit{Nome:}
        \item \textit{Difficoltà:}
        \item \textit{Categoria:}
        \item \textit{Soluzione:}
        \item \textit{Preferito:}
    \end{itemize}
    
    \item Cercare uno o più problemi specifici mediante ricerca filtrata
    per campi.

    \item Selezionare dal catalogo un problema specifico, cosicché la
    descrizione dell'intero problema possa essere visionata per mezzo
    delle funzionalità di cui al \textcolor{blue}{\underbar{\hyperref[seeproblem]{RF \ref*{seeproblem}}}}.
\end{enumerate}
\end{funcreq}

\begin{funcreq}
\label{seeproblem}
\textbf{Consultazione di un problema }
Deve essere fornito un visualizzatore che mostri le informazioni del
problema precedentemente selezionato dal catalogo. La visualizzazione
deve rendere disponibile alla vista dell'utente i contenuti del
problema, specificati nel \textcolor{blue}{\underbar{\hyperref[formatoproblema]{RF \ref*{formatoproblema}}}}.
\end{funcreq}


\begin{funcreq}
\label{formatoproblema}
\textbf{Struttura di un problema }
Tutti i problemi forniti devono possiedere:
\begin{itemize}
    \item Un titolo.

    \item Un testo, scritto prevalentemente in linguaggio naturale,
    che descrive uno scenario che richiede di essere risolto per mezzo
    di un algoritmo. Eventuali immagini e proposizioni matematiche
    formali possono accompagnare i testi.

    \item Almeno 3 esempi di input insieme al relativo output corretto,
    che mostra il risultato atteso.
\end{itemize}
\end{funcreq}

\begin{funcreq}
\label{metadata}
\textbf{Metadati dei problemi } Ogni problema è essenzialmente un'entità
caratterizzata dai seguenti dati descrittivi:
\begin{itemize}
    \item Stato: i problemi già risolti dall'utente sono contrassegnati
    in un campo \textit{stato} apposito.

    \item Nome identificativo: si tratta di una stringa alfanumerica, che
    riprende una o più parole chiave del titolo del problema.
    
    \item Difficoltà: ogni problema possiede un'etichetta (\textit{tag})
    associata che ne valuta indicativamente la difficolta: bassa,
    intermedia e alta.

    \item Categoria: la categoria indica la principale area di interesse
    del problema, come ad esempio le strutture di dati di cui tratta
    (array, grafi, code, ecc.).

    \item Soluzione: un video che mostra come il problema potrebbe essere
    risolto.

    \item Preferito: la marcatura \textit{preferito} è visibile sul problema
    quando esso compare nel catalogo e quando viene consultato.
\end{itemize}
Ai fini della consultazione del catalogo, i filtri applicabili riguardano
i campi \textit{stato}, \textit{difficoltà}, \textit{categoria}, \textit{preferito}
e \textit{nome}.
\end{funcreq}

\subsection{Esercitazione}
\begin{funcreq}
\label{exesession}
\textbf{Avviare una sessione di esercitazione }
L'utente deve poter scegliere dal catalogo il problema desiderato e cominciare
a risolverlo. La sessione di esercitazione deve avvenire in una vista apposita, dove
l'utente deve poter disporre di tutti gli strumenti necessari alla realizzazione
delle seguenti funzionalità:
\begin{enumerate}
    \item Visualizzare il contenuto dell'esercizio (\textcolor{blue}{\underbar{\hyperref[formatoproblema]{RF \ref*{formatoproblema}}}}).
    
    \item Visualizzare i test cases (\textcolor{blue}{\underbar{\hyperref[test]{RF \ref*{test}.2}}}).

    \item Avviare e fermare a proprio piacimento un timer per cronometrare
    il tempo speso per risolvere il quesito scelto.
    
    \item Essere al corrente di quale linguaggio di programmazione sia
    attualmente attivo per la scrittura di codice. Deve altresì essere 
    possibile selezionare uno dei linguaggi messi a disposizione dalla
    piattaforma\footnote{Per approfondimenti sulla disponibilità dei
    linguaggi, si veda \textcolor{blue}{\underbar{\hyperref[scalabilita]{RNF \ref*{scalabilita}.2}}}.}.

    %\item Essere al corrente di quale linguaggio di programmazione sia
    %attualmente attivo per la scrittura di codice. tramite un menu dedicato
    %dal quale deve altresì essere possibile selezionare uno dei linguaggi
    %disponibili\footnote{Per approfondimenti sulla disponibilità dei linguaggi,
    %si veda RNF \ref{scalabilita}.1.}.
    %//(Non dovrebbe far parte dei RNF? Stiamo parlando di UI design, la parte del "menu" sarebbe da incorporare nel testo principale imo)
    
    \item Scrivere, sotto forma di codice nel linguaggio di programmazione
    scelto, l'algoritmo risolutivo del problema attualmente attivo.
\end{enumerate}
\end{funcreq}

%\newpage
\begin{funcreq}
\label{sintax}
\textbf{Correttezza sintattica del codice }
L'utente deve poter verificare che il codice scritto sia corretto e in grando di essere eseguito in maniera appropriata.
Quindi devono essere messe a disposizione le seguenti funzionalità:
\begin{enumerate}
    \item Compilazione del codice.
    \item Visualizzazione di avvisi relativi a eventuali errori di compilazione
    \textit{oppure} di compilazione andata a buon fine. In caso di errori
    di scrittura, l'utente deve poter correggere tali errori riscrivendo
    nell'area destinata al codice.
\end{enumerate}
\end{funcreq}

\begin{funcreq}
\label{test}
\textbf{Verifica della correttezza dell'algoritmo\footnote{La \textit{correttezza} di cui si parla in questo caso riguarda solo l'efficacia risolutiva dell'algoritmo.} }
L'utente deve poter verificare la correttezza del codice scritto eseguendolo
e testandolo:
\begin{enumerate}
    \item Il codice deve essere eseguito sottoponendolo ad un certo numero
    di test cases (al minimo 3),
    cioè fornendo opportune istanze di input (Per esempio,
    se un problema richiede di sommare due numeri interi, il codice risolutivo
    proposto dall'utente verrà eseguito fornendo ad esso coppie di interi
    e registrando i risultati in output.)

    \item L'utente deve poter conoscere l'esito dei test case e, per ognuno
    di questi, l'input e l'output atteso sono visibili.
    Inoltre, l'utente deve poter riscrivere e perfezionare l'algoritmo e
    sottoporre ripetutamente il codice ai test cases.

    \item Nel caso in cui l'utente sia in grado di superare tutti i
    test case disponibili, la sessione di esercitazione deve terminare.
    Deve essere visualizzato il tempo impiegato, se il cronometro è attivo.
\end{enumerate}
\end{funcreq}

\begin{funcreq}
\textbf{Soluzione }
Per ogni problema viene messo a disposizione un video che mostra una
possibile soluzione. Il video è accessibile sia durante la consultazione
del catalogo sia in una sessione di esercitazione.
\end{funcreq}

\begin{center}
    \section*{Utente autenticato}
\end{center}

Nei seguenti requisiti funzionali, si suppone che l'utente appartenga al livello
di accesso \textit{autenticato}, qualora tale informazione dovesse essere omessa.
Si sottolinea nuovamente che l'utente autenticato eredita le stesse funzionalità
accessibili all'utente anonimo.
\\
\subsection{Gestione profilo e account}
\begin{funcreq}
\label{stats}
\textbf{Progressi }
\begin{enumerate}
    \item Il servizio deve prevedere il tracciamento dei progressi dell'utente
    autenticato, memorizzando i riferimenti ai problemi risolti, incrementandone
    il numero ad ogni sessione di esercitazione andata a buon fine. Con riferimento
    al \textcolor{blue}{\underbar{\hyperref[test]{RF \ref*{test}.3}}}, dopo la
    risoluzione, il problema deve inoltre essere contrassegnato come \textit{risolto}
    (nel rispettivo campo \textit{stato}).

    \item L'utente deve poter monitorare i propri progressi, accedendo ai dati
    seguenti:
    \begin{itemize}
        \item Il numero totale di problemi risolti.
        \item Il numero di problemi risolti suddivisi per difficoltà.
    \end{itemize}
\end{enumerate}
\end{funcreq}

\begin{funcreq}
\textbf{Preferiti }
L'utente deve poter contrassegnare un qualsiasi problema come \textit{preferito}.
Tale informazione deve essere visibile nel catalogo, accanto ad ogni problema
contrassegnato in questo modo, unicamente al singolo utente.
\end{funcreq}


\begin{funcreq}
\label{updateaccount}
\textbf{Aggiornamento account }
L'utente registrato e autenticato deve poter modificare i dati chiave del
proprio account in conformità con le opportune procedure di sicurezza:
\begin{enumerate}
\item L'utente deve poter migrare ad un indirizzo email differente. A tale
indirizzo deve essere inviato un messaggio di conferma, previo inserimento
della password. L'email
deve cambiare effettivamente solo nel momento in cui l'utente conferma la
ricezione del messaggio presso il nuovo indirizzo.

\item L'utente deve poter modificare la password del proprio account.
La password può essere cambiata previo inserimento di quella attualmente
associata all'account. La nuova password deve essere digitata due volte
per conferma e deve rispettare quanto specificato dal
\textcolor{blue}{\underbar{\hyperref[legalpassword]{RNF \ref*{legalpassword}}}}. 
L'utente deve essere notificato di tale modifica anche tramite
l'indirizzo email attualmente associato all'account.
\end{enumerate}
\end{funcreq}
    
\begin{funcreq}
\label{logout}
\textbf{Logout }
L'utente autenticato deve poter interrompere la sessione di accesso
al servizio. Questa procedura di \textit{logout} realizza il passaggio dell'utente
dallo stato autenticato a quello anonimo.
\end{funcreq}

\begin{center}
    \section*{Utente amministratore}
\end{center}

Le funzionalità elencate di seguito sono usufruibili esclusivamente dall'utente
amministratore. Inoltre, l'utente amministratore non può accedere alle funzionalità
precedentemente descritte, cioè quelle caratteristiche dei livelli anonimo e autenticato.
\\
\subsection{Gestione del catalogo dei problemi}

\begin{funcreq}
\textbf{Aggiungere un problema }
L'utente amministratore deve poter aggiungere un nuovo problema al catalogo,
fornendo informazioni riguardanti le seguenti caratteristiche:
\begin{enumerate}
    \item Struttura del problema: raccoglie le informazioni utili all'utente
    in fase di consultazione del problema ed esercitazione.
    \begin{itemize}
        \item Titolo.
        \item Testo: scritto prevalentemente in linguaggio naturale,
        descrive uno scenario che richiede di essere risolto per mezzo di
        un algoritmo. Eventuali immagini e proposizioni matematiche possono
        accompagnare i testi.
        \item Almeno tre esempi di input, insieme al relativo output corretto
        che mostra il risultato atteso.
    \end{itemize}

    \item Metadati: rappresentano i campi che descrivono ulteriormente il
    problema. Devono essere visibili nel catalogo.
    \begin{itemize}
        \item 
    \end{itemize}
\end{enumerate}
\end{funcreq}

\begin{funcreq}
\textbf{Modificare un problema }

\end{funcreq}

\begin{funcreq}
\textbf{Eliminare un problema }
\end{funcreq}

\newpage
\section{Requisiti non funzionali}
Vengono ora elencati i requisiti non funzionali (RNF) del servizio.

\subsection{Caratteristiche di sistema}

\begin{nonfuncreq}
\label{scalabilita}
\textbf{Scalabilità }
L'infrastruttura del servizio deve essere scalabile e aperta alle esigenze
derivanti dall'aumento di nuovi utenti. Questo requisito è motivato dalla
disponibilità online del servizio che verrà sviluppato. In particolare:
\begin{enumerate}
    \item L'infrastruttura del servizio deve essere adattabile a eventuali
    crescite nel numero di utenti, in modo da prevenire possibili cali di
    prestazioni eccessivi.

    \item Data l'eterogeneità di linguaggi di programmazione esistenti
    al momento della stesura di questo documento, è importante che il
    servizio sia in grado di accogliere con l'avanzare del tempo codici
    scritti in linguaggi differenti.
\end{enumerate}
\end{nonfuncreq}

\begin{nonfuncreq}
\label{compatibility}
\textbf{Compatibilità }
La piattaforma del servizio deve essere accessibile mediante le versioni
più recenti dei principali browser attualmente disponibili in commercio:
\begin{itemize}
    \item Chrome: 117.0.5938.150
    \item Firefox: 118.0.1
    \item Edge: 117.0.2045.60
\end{itemize}
\end{nonfuncreq}


\begin{nonfuncreq}
\textbf{Usabilità }
La piattaforma del servizio deve permettere all'utente di sfruttare le
funzionalità disponibili al proprio livello di accesso senza l'ausilio di
istruzioni scritte e verbose. L'intuitività dell'interfaccia deve essere
sufficiente a guidare l'utente nella realizzazione dei suoi scopi,
permettendo di imparare almeno il 90\% delle funzionalità del servizio in
meno di 30 minuti. 
\end{nonfuncreq}

\begin{nonfuncreq}
\textbf{Aspetto }
L'interfaccia deve presentarsi gradevole alla vista dell'utente, preferendo
gradazioni cromatiche scure e un contrasto sufficientemente equilibrato,
al fine di garantire la leggibilità e contribuire alla riduzione
dell'affaticamento della vista.
\end{nonfuncreq}

\begin{nonfuncreq}
\textbf{Lingua di sistema }
Il servizio viene erogato in lingua italiana. Altrettanto viene fatto per i
testi dei problemi. I video relativi alla soluzione dei problemi possono
essere in italiano oppure in inglese.
\end{nonfuncreq}

\begin{nonfuncreq}
\textbf{Disponibilità }
Il downtime annuo della piattaforma non deve eccedere gli intervalli di
tempo indispensabili a eventuali opere di manutenzione \textit{sui componenti
interni al servizio}, evitando ove possibile interruzioni di servizio
inutilmente prolungate. Si prevede di non superare un downtime del 2,7\% nel
primo anno (10 giorni su 365, per un totale di 240 ore) dopo il lancio della
piattaforma, per poi mantenere il rapporto al di sotto del 0,85\% (poco più
di 72 ore).

% provare a trovare dati più accurati sul downtime medio dei servizi google
L'affidabilità e la disponibilità dei servizi terzi, che permettono alla
piattaforma di funzionare (database, contenuti multimediali e altro),
permettono di constatare con sicurezza che eventuali interruzioni non dovute
alla piattaforma in sé contribuiscono in minima parte ai dati stimati
sopra (si tratta di ordini inferiori alla decina di ore).
\end{nonfuncreq}

\begin{nonfuncreq}
\textbf{Prestazioni }
L'esperienza di utilizzo deve poter essere soddisfacente in relazione
ai livelli di prestazioni dei moderni siti web, purché l'utente utilizzi
il servizio in condizioni di connettività sufficienti. In particolare,
i tempi di reazione relativi a caricamenti e transizioni tra pagine non
deve eccedere i 2 secondi.
\end{nonfuncreq}

\subsection{Privacy e sicurezza}

\begin{nonfuncreq}
\textbf{Privacy e trattamento dei dati }
Il servizio deve essere progettato e realizzato in ottemperanza delle
vigenti disposizioni di legge in materia di tutela della privacy e
trattamento dei dati:
\begin{enumerate}
    \item L'applicazione fornita dal servizio deve essere conforme
    al regolamento \href{https://www.garanteprivacy.it/documents/10160/0/Regolamento+UE+2016+679.+Arricchito+con+riferimenti+ai+Considerando+Aggiornato+alle+rettifiche+pubblicate+sulla+Gazzetta+Ufficiale++dell%27Unione+europea+127+del+23+maggio+2018}{\textcolor{blue}{\underbar{UE n.2016/679}}} (GDPR) per la protezione dei dati.
\end{enumerate}
\end{nonfuncreq}

\begin{nonfuncreq}
\textbf{Connessione sicura }
La comunicazione con il client deve essere protetta da protocolli
di sicurezza, come \texttt{https}, che consentano di preservare la
riservatezza dei dati scambiati tra piattaforma e client utente.
\end{nonfuncreq}

\begin{nonfuncreq}
\label{legalpassword}
\textbf{Password strength }
In tutti gli scenari nei quali è richiesto l'inserimento di una password,
devono essere rispettate le seguenti caratteristiche:
\begin{itemize}
    \item Lunghezza compresa tra 8 e 64 caratteri.
    \item Contenere almeno una lettera maiuscola.
    \item Contenere almeno una lettera minuscola.
    \item Contenere almeno un numero.
    \item Contenere almeno un carattere speciale scelto tra i
    seguenti:
    \begin{center}
        \verb|! ? # $ % & @ * + . , ; : / - = _ \ ( ) [ ] { }|
    \end{center}
\end{itemize}
\end{nonfuncreq}

%\begin{nonfuncreq}
%\textbf{Integrità e operazioni su dati sensibili }
%L'integrità di dati sensibili dell'account, quali email ma soprattutto password,
%è garantita da opportune procedure di conferma e notifica all'utente.
%Data la natura di queste operazioni di sicurezza, i requisiti funzionali
%relativi alla gestione dei dati sensibili (RF \ref{signup}, RF \ref{savepassword},
%RF \ref{updateaccount}) provvedono a descrivere il comportamento del
%servizio nel caso di errori e incongruenze.
%\end{nonfuncreq}



\newpage
\section{Design front-end}
In questa sezione vengono presentati alcuni mock-up dell'interfaccia che
la piattaforma online espone all'utente.

\begin{frontend}
\textbf{Pagina di login }
\end{frontend}
\begin{figure}[H]
\centering
\includegraphics[scale=0.22]{materiale/immaginife/login.jpeg}
\end{figure}

\begin{frontend}
\textbf{Pagina di registrazione }
\end{frontend}
\begin{figure}[H]
\centering
\includegraphics[scale=0.22]{materiale/immaginife/registrazione.jpeg}
\end{figure}

\newpage
\begin{frontend}
\textbf{Home page }
Il catalogo e il profilo utente, insieme a tutti i collegamenti che consentono
di usufruire della maggior parte delle funzionalità del servizio, sono
contenute in una pagina principale.
\end{frontend}
\begin{figure}[H]
\centering
\includegraphics[scale=0.195]{materiale/immaginife/homecatalogo.jpeg}
\end{figure}

\begin{frontend}
\textbf{Pagina di esercitazione }

\end{frontend}
\begin{figure}[H]
\centering
\includegraphics[scale=0.195]{materiale/immaginife/esercitazione.jpeg}
\end{figure}

\begin{frontend}
\textbf{Logout }
Un pulsante di logout è raggiungibile in ogni momento da parte dell'utente
autenticato.
\end{frontend}
\begin{figure}[H]
\centering
\includegraphics[scale=0.2]{materiale/immaginife/logout.jpeg}
\end{figure}

\newpage
\section{Design back-end}
Nella sezione seguente vengono elencati i servizi con i quali la piattaforma
interagisce per ottenere supporto alle sue funzionalità.

\begin{backend}
\textbf{Servizio di autenticazione Google }
L'utente potrà scegliere di autenticarsi mediante un account Google.
\end{backend}

\begin{backend}
\textbf{Contenuti multimediali }
Per reperire e riprodurre i video contenenti le soluzioni dei problemi,
il servizio si affiderà alla piattaforma YouTube.
\end{backend}

\begin{backend}
\textbf{Database }
Verrà impiegato il servizio di database Firebase per la memorizzazione
degli utenti, dei problemi del catalogo e di tutti i dati associati.
\end{backend}
% alternativa: mongoDB

\begin{backend}
\textbf{Notifica via posta elettronica }
L'invio dei messaggi email (da parte della piattaforma) sarà gestito da servizi di terze parti.
\end{backend}
% possibili candidati: gmail

\begin{figure}[H]
\centering
\includegraphics[scale=0.35]{materiale/backend.jpg}
\end{figure}

\newpage
\section{Cronologia}
Nella cronologia sono riportate le versioni del documento
\begin{itemize}
    \item \textit{Versione 1.0:} prima stesura definitiva.
    \item \textit{Versione 2.0:} definizione dei livelli di accesso e conseguente estensione delle funzionalità.
    \item \textit{Versione 2.1:} ridefinizione delle funzionalità del livello amministratore.
\end{itemize}

\end{document}