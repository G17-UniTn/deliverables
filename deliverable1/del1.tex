\documentclass[11pt, a4paper]{article}

%\usepackage[T1]{fontenc}
\usepackage[utf8]{inputenc} % comment when using lualatex
\usepackage[italian]{babel}
\usepackage{fullpage}
\usepackage{graphicx}
\usepackage[hidelinks]{hyperref,xcolor} % link di pagina
\renewcommand\UrlFont{\color{blue}\rmfamily}



\title{Analisi dei Requisiti}

\author{Raffaele \textsc{Castagna}\\
Alberto \textsc{Rovesti}\\
Zeno \textsc{Saletti}}

\newcommand{\groupNumber}{G17}

% Web address for the project (if any)
% \newcommand{\homepage}{\url{https://www.ntnu.edu/studies/courses/IT3010}}

% data
\date{\today}

\makeatletter{}

% IL PREAMBOLO FINISCE QUI %%%%%%%%%%%%%%%%%%%%%%%%%%%%%%%%%%%%%%%%%%%%%%%%%%%%






\begin{document}

% La pagina di copertina si trova in un file .tex a parte
\begin{titlepage}
\newcommand{\HRule}{\rule{\linewidth}{0.3mm}} % Defines a new command for horizontal lines, change thickness here
\center % Centre everything on the page

%------------------------------------------------
%	Logo
%------------------------------------------------
\includegraphics[width=0.3\textwidth]{materiale/UniTrento_logo_ITA_colore.png}\\[0.5cm]
%------------------------------------------------
%	Headings
%------------------------------------------------
\textsc{\Large Dipartimento di Ingegneria\\e Scienza dell'Informazione}\\[1.5cm]

{\Huge\textbf{Sleep Code}}\\[0.5cm]
\textsc{\large Progetto per il Corso di Ingegneria del Software}\\
\textsc{\large Anno Accademico 2023-2024}\\[0.5cm]

%------------------------------------------------
%	Title
%------------------------------------------------

\HRule\\[0.4cm]
{\huge\bfseries \@title}\\[0.1cm]
\HRule\\[1cm]

\begin{minipage}{\textwidth}
\begin{flushleft}
\textit{Descrizione:} documento di analisi dei requisiti funzionali, non funzionali, front-end e back-end.
\end{flushleft}
\end{minipage}\\[1.5cm]


\begin{minipage}{0.4\textwidth}
\begin{flushleft}
\large
\textit{Numero documento:} D1
\end{flushleft}
\end{minipage}
\begin{minipage}{0.4\textwidth}
\begin{flushright}
\large
\textit{Versione documento:} 2.4
\end{flushright}
\end{minipage}\\[1.5cm]

%------------------------------------------------
%	Author(s)
%------------------------------------------------
\begin{minipage}{0.4\textwidth}
\begin{flushleft}
\large
\textit{Membri del gruppo:}\\
\@author % Your name
\end{flushleft}
\end{minipage}
~
\begin{minipage}{0.4\textwidth}
\begin{flushright}
\large
\textit{Numero gruppo: }
\groupNumber
\end{flushright}
\end{minipage}

% 	If you don't want a supervisor, uncomment the two lines below and comment the code above
% 	{\large\textit{Author(s)}}\\
% 	\@author % Your name

%------------------------------------------------
%	Date
%------------------------------------------------

\vfill\vfill
\textit{Ultima revisione:}
{\@date}

\end{titlepage}

\tableofcontents


\newpage
\section{Introduzione}

\subsection{Obiettivo del progetto}
Il progetto proposto si prefigge, come scopo fondante, di fornire alla comunità
di giovani informatici un servizio di \textit{esercitazione} e di
\textit{raccolta} di problemi mirati alla programmazione e alla progettazione
di piccoli algoritmi risolutivi, mediante la scrittura di codice.

\subsection{Contesto di business?}



\subsection{Attori coinvolti}
Per comprendere meglio i requisiti che verranno descritti in seguito (in
particolar modo quelli funzionali), è innanzitutto essenziale specificare
il pubblico, insieme alle loro potenziali esigenze, al quale il servizio
intende rivolgersi.
Tale servizio vuole rendersi utile soprattutto a coloro che sono coinvolti
in percorsi di studio attinenti all'ambito informatico, ma specialmente anche
a chiunque desideri cimentarsi nella risoluzione di piccoli problemi di
programmazione; pertanto ci si aspetta che chiunque desideri usufruire del
servizio possieda . Il progetto che verrà sviluppato ha di fatto lo scopo di
offrire

D'ora in avanti, in questo e nei successivi documenti, questo pubblico di
singoli individui appena descritti verrà indicato con il termine
\textit{utenti}.


\section*{Note}

il progetto vuole proporsi come piattaforma per
l'esercitazione, l'approfondimento e l'autovalutazione delle proprie conoscenze
e abilità di \textit{problem solving} legate alla programmazione e scrittura di
codice.

Per questo motivo, è essenziale che il servizio si renda utile alla
raccolta di quesiti e risorse da offrire agli utenti che desiderano usufruirne
per realizzare gli scopi di cui sopra. Per risorse si intendono: i sopracitati
problemi, i quali richiedono una soluzione codificata in forma di algoritmo;
gli strumenti utili alla ricerca e consultazione dei
quesiti\footnote{In questo documento, intendiamo quesiti e problemi
come termini per indicare la stessa cosa.}, alla scrittura e
compilazione di codice (sulla base dei linguaggi di programmazione
messi a disposizione), all'esecuzione e alla valutazione della
correttezza dell'algoritmo, scritto per mezzo del codice, mediante
opportune operazioni di testing automatizzate.

\section{Requisiti funzionali}
Vengono di seguito elencati i principali requisiti funzionali della
piattaforma.



\section{Requisiti non funzionali}



\section{Design front-end}
\section{Design back-end}

\end{document}
