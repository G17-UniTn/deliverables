% COMMENTO TOLTO %%%%%%%%%%%%%%%%%%%%%%
\documentclass[11pt, a4paper]{article}

%\usepackage[T1]{fontenc}
\usepackage[utf8]{inputenc} % comment when using lualatex
\usepackage[italian]{babel}

\usepackage{babel}
\usepackage{fullpage}
\usepackage{graphicx}
\usepackage[hidelinks]{hyperref,xcolor}
\renewcommand\UrlFont{\color{blue}\rmfamily}


% Title of your project
\title{Analisi dei Requisiti}

% Group names(s)
\author{Raffaele \textsc{Castagna}\\
Alberto \textsc{Rovesti}\\
Zeno \textsc{Saletti}}

% Group number
\newcommand{\groupNumber}{G17}

% Any comments for us
\newcommand{\comments}{Comments for teachers of the course}

% Web address for the project (if any)
% \newcommand{\homepage}{\url{https://www.ntnu.edu/studies/courses/IT3010}}

% Date for title page, default is today and 
\date{\today}

\makeatletter{}

% PREAMBLE ENDS HERE %%%%%%%%%%%%%%%%%%%%%%%%%%%%%%%%%%%%%%%%%%%%%%%%%%%%%%

\begin{document}

\begin{titlepage}
\newcommand{\HRule}{\rule{\linewidth}{0.3mm}} % Defines a new command for horizontal lines, change thickness here
\center % Centre everything on the page


%------------------------------------------------
%	Logo
%------------------------------------------------
\includegraphics[width=0.3\textwidth]{./unioftrento.png}\\[0.5cm]
%------------------------------------------------
%	Headings
%------------------------------------------------
\textsc{\Large Dipartimento di Ingegneria e Sceienza dell'Informazione}\\[1.5cm]

{\Huge\textbf{Sleep Code}}\\[0.5cm]
\textsc{\large Progetto per il Corso di Ingegneria del Software}\\[0.5cm]

%------------------------------------------------
%	Title
%------------------------------------------------

\HRule\\[0.4cm]
{\huge\bfseries \@title}\\[0.1cm]
\HRule\\[1cm]

\begin{minipage}{\textwidth}
\begin{flushleft}
\textit{Descrizione:} documento di analisi dei requisiti funzionali, non funzionali, front-end e back-end.\\
\textit{Numero documento:} D1
\end{flushleft}
\end{minipage}\\[1.5cm]



%------------------------------------------------
%	Author(s)
%------------------------------------------------
\begin{minipage}{0.4\textwidth}
\begin{flushleft}
\large
\textit{Membri del gruppo:}\\
\@author % Your name
\end{flushleft}
\end{minipage}
~
\begin{minipage}{0.4\textwidth}
\begin{flushright}
\large
\textit{Numero gruppo: }
\groupNumber
\end{flushright}
\end{minipage}

% 	If you don't want a supervisor, uncomment the two lines below and comment the code above
% 	{\large\textit{Author(s)}}\\
% 	\@author % Your name

%------------------------------------------------
%	Date
%------------------------------------------------

\vfill\vfill
\textit{Ultima revisione:}
{\@date}
\vfill\vfill\vfill

\footnotesize{Comments: \comments}


\end{titlepage}

\tableofcontents

\newpage
\section{Introduzione e scopi}
\subsection{Obiettivi del progetto}


\subsection{Premessa}
Riteniamo sia inadeguato esporre gli obiettivi del progetto senza innanzitutto specificare a \textit{chi} tale progetto intende rivolgersi

\subsection{Obiettivi}
Il progetto proposto si prefigge, come scopo fondante, di fornire alla comunità
di giovani informatici un servizio di \textit{esercitazione} e di
\textit{raccolta} di problemi mirati alla programmazione.

Tale servizio si rivolge, in modo particolare, a coloro che sono coinvolti
in percorsi di studio attinenti all'ambito informatico, ma anche a chiunque
desideri cimentarsi nella risoluzione di piccoli problemi di programmazione.
A questo pubblico di singoli individui, che da qui in poi verranno indicati
col termine \textit{utenti}, il progetto vuole proporsi come piattaforma per l'esercitazione, l'approfondimento e l'autovalutazione delle proprie conoscenze e abilità di \textit{problem solving}
legate alla programmazione e scrittura di codice. Per questo motivo, è essenziale che il servizio
si renda utile alla raccolta di quesiti e risorse da offrire agli utenti
che desiderano usufruirne per realizzare gli scopi di cui sopra.
Per risorse si intendono: i sopracitati problemi, i quali richiedono
una soluzione codificata in forma di algoritmo;
gli strumenti utili alla ricerca e consultazione dei
quesiti\footnote{In questo documento, intendiamo quesiti e problemi
come termini per indicare la stessa cosa.}, alla scrittura e
compilazione di codice (sulla base dei linguaggi di programmazione
messi a disposizione), all'esecuzione e alla valutazione della
correttezza dell'algoritmo, scritto per mezzo del codice, mediante
opportune operazioni di testing automatizzate.


\section{Requisiti funzionali}
Il progetto consiste in una piattaforma che dovrà essere accessibile online, da parte dei singoli utenti, mediante credenziali di login.

\section{Requisiti non funzionali}



\section{Design front-end}
\section{Design back-end}


\end{document}
