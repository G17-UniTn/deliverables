\documentclass[11pt, a4paper]{article}

%\usepackage[T1]{fontenc}
\usepackage[utf8]{inputenc} % comment when using lualatex
\usepackage[italian]{babel}
\usepackage{fullpage}
\usepackage{graphicx}
\usepackage[hidelinks]{hyperref,xcolor} % link di pagina
\usepackage[bottom]{footmisc} % note appiccicate al fondo della pagina

\renewcommand\UrlFont{\color{blue}\rmfamily}

\usepackage{amsthm}
\theoremstyle{definition}

\newtheorem{funcreq}{RF} %% numerazione dei requisiti funzionali
\newtheorem{nonfuncreq}{RNF} %% requisiti non funzionali


\title{Analisi dei Requisiti}

\author{Raffaele \textsc{Castagna}\\
Alberto \textsc{Rovesti}\\
Zeno \textsc{Saletti}}

\newcommand{\groupNumber}{G17}

% Web address for the project (if any)
% \newcommand{\homepage}{\url{https://www.ntnu.edu/studies/courses/IT3010}}

% data
\date{\today}

\makeatletter{}

% IL PREAMBOLO FINISCE QUI %%%%%%%%%%%%%%%%%%%%%%%%%%%%%%%%%%%%%%%%%%%%%%%%%%%%






\begin{document}

% La pagina di copertina si trova in un file .tex a parte
\begin{titlepage}
\newcommand{\HRule}{\rule{\linewidth}{0.3mm}} % Defines a new command for horizontal lines, change thickness here
\center % Centre everything on the page

%------------------------------------------------
%	Logo
%------------------------------------------------
\includegraphics[width=0.3\textwidth]{materiale/UniTrento_logo_ITA_colore.png}\\[0.5cm]
%------------------------------------------------
%	Headings
%------------------------------------------------
\textsc{\Large Dipartimento di Ingegneria\\e Scienza dell'Informazione}\\[1.5cm]

{\Huge\textbf{Sleep Code}}\\[0.5cm]
\textsc{\large Progetto per il Corso di Ingegneria del Software}\\
\textsc{\large Anno Accademico 2023-2024}\\[0.5cm]

%------------------------------------------------
%	Title
%------------------------------------------------

\HRule\\[0.4cm]
{\huge\bfseries \@title}\\[0.1cm]
\HRule\\[1cm]

\begin{minipage}{\textwidth}
\begin{flushleft}
\textit{Descrizione:} documento di analisi dei requisiti funzionali, non funzionali, front-end e back-end.
\end{flushleft}
\end{minipage}\\[1.5cm]


\begin{minipage}{0.4\textwidth}
\begin{flushleft}
\large
\textit{Numero documento:} D1
\end{flushleft}
\end{minipage}
\begin{minipage}{0.4\textwidth}
\begin{flushright}
\large
\textit{Versione documento:} 2.4
\end{flushright}
\end{minipage}\\[1.5cm]

%------------------------------------------------
%	Author(s)
%------------------------------------------------
\begin{minipage}{0.4\textwidth}
\begin{flushleft}
\large
\textit{Membri del gruppo:}\\
\@author % Your name
\end{flushleft}
\end{minipage}
~
\begin{minipage}{0.4\textwidth}
\begin{flushright}
\large
\textit{Numero gruppo: }
\groupNumber
\end{flushright}
\end{minipage}

% 	If you don't want a supervisor, uncomment the two lines below and comment the code above
% 	{\large\textit{Author(s)}}\\
% 	\@author % Your name

%------------------------------------------------
%	Date
%------------------------------------------------

\vfill\vfill
\textit{Ultima revisione:}
{\@date}

\end{titlepage}

\tableofcontents


\newpage
\section{Introduzione}
\subsection{Scopo del documento}
Le informazioni contenute in questo documento concorrono ad esporre l'analisi
dei requisiti relativa al progetto \textit{SleepCode}. In particolare, dopo
aver specificato gli obiettivi e gli attori coinvolti—utenti finali e
utilizzatori del frutto di questo progetto—verranno definiti i requisiti
funzionali e non funzionali; verrà presentata una proposta di design di
back-end; infine saranno riportati i servizi di back-end.


\subsection{Obiettivo del progetto}
Il progetto proposto si prefigge, come scopo fondante, di fornire alla comunità
di giovani informatici un servizio online di \textit{esercitazione} e di
\textit{raccolta} di problemi mirati alla programmazione e alla progettazione
di piccoli algoritmi risolutivi, mediante la scrittura di codice.

\subsection{Attori coinvolti ed esigenze}
Per comprendere meglio i requisiti che verranno descritti in seguito (in
particolar modo quelli funzionali), è innanzitutto essenziale specificare
il pubblico, insieme alle loro potenziali esigenze, al quale il servizio
intende rivolgersi.
Tale servizio vuole rendersi utile soprattutto a coloro che sono coinvolti
in percorsi di studio attinenti all'ambito informatico, ma specialmente anche
a chiunque desideri cimentarsi nella risoluzione di piccoli problemi di
programmazione; pertanto ci si aspetta che chiunque desideri usufruire del
servizio possieda almeno le conoscenze basilari della programmazione. Esempi
di queste nozioni pregresse, che tuttavia non devono necessariamente essere ampie e approfondite per utilizzare il servizio\footnote{Gli utenti più esperti possono indubbiamente trarre vantaggio dal loro bagaglio culturale per
approcciarsi con maggior facilità al servizio.}, sono: cosa si intende per
algoritmo e linguaggio di programmazione, familiarità nell'uso di qualche
linguaggio di programmazione, tipi e strutture di dati più comuni.
%% PROBLEMA: il servizio NON offre TUTTI i linguaggi di programmazione
%% esistenti, quindi l'utente potrebbe essere limitato sotto questo
%% punto di vista.

Di fatto, il progetto che verrà sviluppato ha come scopo principale di
creare una piattaforma accessibile a singoli utilizzatori che desiderano
esercitarsi, valutare e approfondire le personali conoscenze e abilità di
\textit{problem solving} legate alla programmazione.
D'ora in avanti, in questo e nei successivi documenti, questo pubblico di
individui appena descritti verrà indicato con il termine \textit{utenti}.

\newpage
\section{Requisiti funzionali}
Vengono di seguito elencati i principali requisiti funzionali del progetto.
Essi sono organizzati secondo uno schema che segue gli obiettivi elencati nei
paragrafi precedenti. Più in particolare, ogni sottosezione di questa parte
del documento risponde a diversi scopi precedentemente accennati, suddividendo
eventuali macro-funzioni in requisiti minori nel caso di obiettivi più ampi.

\subsection*{Accesso al servizio}

\begin{funcreq}
Altro requisito funzionale
\end{funcreq}


\subsection*{Consultazione dei problemi}
Il servizio deve mettere a disposizione, tra le altre risorse, un insieme di problemi sui quali l'utente possa esercitarsi. [come sono fatti i problemi,
come sono classificati, come vengono ordinati e categorizzati.]

\begin{funcreq}
L'utente deve poter consultare un catalogo di problemi e navigare al suo
interno. In particolare, l'utente deve poter:
\begin{enumerate}
    \item Visualizzare tale catalogo in una vista dedicata (si veda la
    sezione riguardante il design di front-end per ulteriori dettagli).
    \item Cercare uno o più problemi specifici mediante ricerca per campi (nome,
    difficoltà).
\end{enumerate}
\end{funcreq}

\subsection*{Esercitazione e risoluzione dei problemi}
\begin{funcreq}
L'utente deve poter selezionare, attraverso l'apposito catalogo, il
problema desiderato avviando una sessione di esercitazione con lo scopo
di risolverlo. A tal fine, l'utente deve poter:
\begin{enumerate}
    \item Attivare il problema scelto, cioè visualizzarlo ed eseguire le
    funzionalità mostrate nei prossimi punti.
    
    \item Essere al corrente di quale linguaggio di programmazione sia
    attualmente attivo per la scrittura di codice, tramite un menu dedicato
    dal quale deve altresì essere possibile selezionare uno dei linguaggi
    disponibili\footnote{Per approfondimenti sulla disponibilità dei linguaggi, si veda RNF \ref{scalabilita}}.
    
    \item Scrivere, sotto forma di codice nel linguaggio di programmazione
    scelto, l'algoritmo risolutivo del problema attualmente attivo. La
    scrittura deve poter essere effettuata in una vista o finestra apposita.
\end{enumerate}
\end{funcreq}

\begin{funcreq}
L'utente deve poter verificare che il codice scritto sia sintatticamente
corretto e pronto per l'esecuzione. Quindi devono essere messe a disposizione
le seguenti funzionalità:
\begin{enumerate}
    \item Compilazione del codice.
    \item Visualizzazione di avvisi relativi a eventuali errori di sintassi
    \textit{oppure} di compilazione andata a buon fine.
\end{enumerate}
\end{funcreq}

\begin{funcreq}
L'utente deve poter verificare la correttezza del codice scritto eseguendolo,
ovvero:
\begin{enumerate}
    \item .
\end{enumerate}
\end{funcreq}

\subsection*{Valutazione della propria carriera}




\section{Requisiti non funzionali}

\begin{nonfuncreq}
\label{scalabilita}
\textbf{Scalabilità:}
\end{nonfuncreq}

\section{Design front-end}

\section{Design back-end}

\end{document}
