\documentclass[11pt, a4paper]{article}

%\usepackage[T1]{fontenc}
\usepackage[utf8]{inputenc} % comment when using lualatex
\usepackage[italian]{babel}
\usepackage{fullpage}
\usepackage{graphicx}
\usepackage[hidelinks]{hyperref,xcolor} % link di pagina
\usepackage[bottom]{footmisc} % note appiccicate al fondo della pagina

\renewcommand\UrlFont{\color{blue}\rmfamily}

\usepackage{amsthm}
\theoremstyle{definition}

\newtheorem{funcreq}{RF} %% numerazione dei requisiti funzionali
\newtheorem{nonfuncreq}{RNF} %% requisiti non funzionali


\title{Analisi dei Requisiti}

\author{Raffaele \textsc{Castagna}\\
Alberto \textsc{Rovesti}\\
Zeno \textsc{Saletti}}

\newcommand{\groupNumber}{G17}

% Web address for the project (if any)
% \newcommand{\homepage}{\url{https://www.ntnu.edu/studies/courses/IT3010}}

% data
\date{\today}

\makeatletter{}

% IL PREAMBOLO FINISCE QUI %%%%%%%%%%%%%%%%%%%%%%%%%%%%%%%%%%%%%%%%%%%%%%%%%%%%






\begin{document}

% La pagina di copertina si trova in un file .tex a parte
\begin{titlepage}
\newcommand{\HRule}{\rule{\linewidth}{0.3mm}} % Defines a new command for horizontal lines, change thickness here
\center % Centre everything on the page

%------------------------------------------------
%	Logo
%------------------------------------------------
\includegraphics[width=0.3\textwidth]{materiale/UniTrento_logo_ITA_colore.png}\\[0.5cm]
%------------------------------------------------
%	Headings
%------------------------------------------------
\textsc{\Large Dipartimento di Ingegneria\\e Scienza dell'Informazione}\\[1.5cm]

{\Huge\textbf{Sleep Code}}\\[0.5cm]
\textsc{\large Progetto per il Corso di Ingegneria del Software}\\
\textsc{\large Anno Accademico 2023-2024}\\[0.5cm]

%------------------------------------------------
%	Title
%------------------------------------------------

\HRule\\[0.4cm]
{\huge\bfseries \@title}\\[0.1cm]
\HRule\\[1cm]

\begin{minipage}{\textwidth}
\begin{flushleft}
\textit{Descrizione:} documento di analisi dei requisiti funzionali, non funzionali, front-end e back-end.
\end{flushleft}
\end{minipage}\\[1.5cm]


\begin{minipage}{0.4\textwidth}
\begin{flushleft}
\large
\textit{Numero documento:} D1
\end{flushleft}
\end{minipage}
\begin{minipage}{0.4\textwidth}
\begin{flushright}
\large
\textit{Versione documento:} 2.4
\end{flushright}
\end{minipage}\\[1.5cm]

%------------------------------------------------
%	Author(s)
%------------------------------------------------
\begin{minipage}{0.4\textwidth}
\begin{flushleft}
\large
\textit{Membri del gruppo:}\\
\@author % Your name
\end{flushleft}
\end{minipage}
~
\begin{minipage}{0.4\textwidth}
\begin{flushright}
\large
\textit{Numero gruppo: }
\groupNumber
\end{flushright}
\end{minipage}

% 	If you don't want a supervisor, uncomment the two lines below and comment the code above
% 	{\large\textit{Author(s)}}\\
% 	\@author % Your name

%------------------------------------------------
%	Date
%------------------------------------------------

\vfill\vfill
\textit{Ultima revisione:}
{\@date}

\end{titlepage}

\tableofcontents


\newpage
\section{Introduzione e scopi}
\subsection{Scopo del documento}
Le informazioni contenute in questo documento concorrono ad esporre l'analisi
dei requisiti relativa al progetto \textit{SleepCode}. In particolare, dopo
aver specificato gli obiettivi e gli attori coinvolti—utenti finali e
utilizzatori del frutto di questo progetto—verranno definiti i requisiti
funzionali e non funzionali; verrà presentata una proposta di design di
back-end; infine saranno riportati i servizi di back-end.


\subsection{Obiettivo del progetto}
Il progetto proposto si prefigge, come scopo fondante, di fornire alla comunità
di giovani informatici un servizio online di \textit{esercitazione} e di
\textit{raccolta} di problemi mirati alla programmazione e alla progettazione
di piccoli algoritmi risolutivi, mediante la scrittura di codice.

\subsection{Attori coinvolti ed esigenze}
Per comprendere meglio i requisiti che verranno descritti in seguito (in
particolar modo quelli funzionali), è innanzitutto essenziale specificare
il pubblico, insieme alle loro potenziali esigenze, al quale il servizio
intende rivolgersi.
Tale servizio vuole rendersi utile soprattutto a coloro che sono coinvolti
in percorsi di studio attinenti all'ambito informatico, ma specialmente anche
a chiunque desideri cimentarsi nella risoluzione di piccoli problemi di
programmazione; pertanto ci si aspetta che chiunque desideri usufruire del
servizio possieda almeno le conoscenze basilari della programmazione. Esempi
di queste nozioni pregresse, che tuttavia non devono necessariamente essere ampie e approfondite per utilizzare il servizio\footnote{Gli utenti più esperti possono indubbiamente trarre vantaggio dal loro bagaglio culturale per
approcciarsi con maggior facilità al servizio.}, sono: cosa si intende per
algoritmo e linguaggio di programmazione, familiarità nell'uso di qualche
linguaggio di programmazione, tipi e strutture di dati più comuni.
%% PROBLEMA: il servizio NON offre TUTTI i linguaggi di programmazione
%% esistenti, quindi l'utente potrebbe essere limitato sotto questo
%% punto di vista.

Di fatto, il progetto che verrà sviluppato ha come scopo principale di
creare una piattaforma accessibile a singoli utilizzatori che desiderano
esercitarsi, valutare e approfondire le personali conoscenze e abilità di
\textit{problem solving} legate alla programmazione.
D'ora in avanti, in questo e nei successivi documenti, questo pubblico di
individui appena descritti verrà indicato con il termine \textit{utenti}.

\newpage
\section{Requisiti funzionali}
Vengono di seguito elencati i principali requisiti funzionali (RF)
del progetto. Essi sono organizzati secondo uno schema che segue gli
obiettivi elencati nei paragrafi precedenti. Più in particolare,
ogni sottosezione di questa parte del documento risponde a diversi
scopi precedentemente accennati, suddividendo eventuali macro-funzioni
in requisiti minori nel caso di obiettivi di più ampia portata.

\subsection{Accesso}

\begin{funcreq}
Altro requisito funzionale
\end{funcreq}


\subsection{Consultazione dei problemi}

\begin{funcreq}
\textbf{Consultazione del catalogo dei problemi:}
Il servizio deve mettere a disposizione un insieme di problemi sui quali
l'utente possa esercitarsi. L'utente deve poter consultare un catalogo,
atto a raccogliere i quesiti, e navigare al suo interno. Deve quindi
essere possibile:
\begin{enumerate}
    \item Visualizzare tale catalogo in una vista dedicata.
    
    \item Cercare uno o più problemi specifici mediante ricerca filtrata
    per campi, specificati dai metadati elencati al RNF \ref{metadata},
    oppure non filtrata.

    \item Selezionare dal catalogo un problema specifico, cosicché la
    descrizione dell'intero problema possa essere visionata per mezzo
    delle funzionalità di cui al RF \ref{seeproblem}.
\end{enumerate}
\end{funcreq}

\begin{funcreq}
\label{seeproblem}
\textbf{Consultazione di un problema:}
Deve essere fornito un visualizzatore del documento contenente le
informazioni del singolo problema precedentemente selezionato dal
catalogo. La visualizzazione deve rendere disponibile alla vista
dell'utente i dati relativi al problema e specificati nel RNF
\ref{formatoproblema}.
\end{funcreq}

\subsection{Esercitazione}
\begin{funcreq}
\textbf{Avviare una sessione di esercitazione:}
L'utente deve poter selezionare, attraverso l'apposito catalogo, il
problema desiderato avviando una sessione di esercitazione con lo scopo
di risolverlo. A tal fine, l'utente deve poter:
\begin{enumerate}
    \item Attivare, dopo averlo visualizzato, il problema scelto. L'attivazione permette di accedere alle funzionalità descritte nei prossimi punti, oltre a incrementare di una unità il numero di tentativi
    effettuati dall'utente su quel problema (a meno che il problema non
    sia già stato risolto in precedenza). L'avvio della sessione di
    esercitazione deve avvenire previa conferma da parte dell'utente.
    
    \item Essere al corrente di quale linguaggio di programmazione sia
    attualmente attivo per la scrittura di codice, tramite un menu dedicato
    dal quale deve altresì essere possibile selezionare uno dei linguaggi
    disponibili\footnote{Per approfondimenti sulla disponibilità dei linguaggi, si veda RNF \ref{scalabilita}.1.}.
    
    \item Scrivere, sotto forma di codice nel linguaggio di programmazione
    scelto, l'algoritmo risolutivo del problema attualmente attivo. La
    scrittura deve poter essere effettuata in una vista o finestra apposita.
\end{enumerate}
\end{funcreq}

\begin{funcreq}
\textbf{Correttezza sintattica del codice:}
L'utente deve poter verificare che il codice scritto sia sintatticamente
corretto e pronto per l'esecuzione. Quindi devono essere messe a disposizione
le seguenti funzionalità:
\begin{enumerate}
    \item Compilazione del codice.
    \item Visualizzazione di avvisi relativi a eventuali errori di sintassi
    \textit{oppure} di compilazione andata a buon fine. In caso di errori
    di scrittura, l'utente deve poter correggere tali errori riscrivendo
    nell'area destinata al codice.
\end{enumerate}
\end{funcreq}

\begin{funcreq}
\textbf{Verifica della correttezza\footnote{La \textit{correttezza} di cui si parla in questo caso riguarda solo l'efficacia risolutiva dell'algoritmo.}:}
L'utente deve poter verificare la correttezza del codice scritto eseguendolo
e testandolo:
\begin{enumerate}
    \item Il codice deve essere eseguito sottoponendolo ad un certo numero
    di test cases (al minimo 3),
    cioè fornendo opportune istanze di input (Per esempio,
    se un problema richiede di sommare due numeri interi, il codice risolutivo
    proposto dall'utente verrà eseguito fornendo ad esso coppie di interi
    e registrando i risultati in output.)

    \item L'utente deve poter verificare quali e quanti test cases sono andati
    a buon fine. Per ogni test case, solo le istanze in input potranno
    essere visualizzate. Inoltre, l'utente deve poter riscrivere e
    perfezionare l'algoritmo e sottoporre ripetutamente il codice ai
    test cases.

    \item L'utente deve poter terminare la sessione in un tempo finito.
    Pertanto: nel caso in cui tutti i test cases sono andati a buon fine
    la sessione di esercitazione deve terminare e la carriera dell'utente
    deve essere aggiornata di conseguenza, quindi contrassegnando come
    \textit{risolto} il problema corrente insieme al numero di
    tentativi impiegati fino all'istante della risoluzione; in caso contrario,
    l'utente deve avere la possibilità di rinunciare a fornire soluzioni
    al problema, terminando la sessione di esercitazione.
\end{enumerate}
\end{funcreq}

\subsection{Gestione profilo}
Si intende per \textit{carriera} l'insieme dei dati estratti dai tentativi
effettuati dal singolo utente durante le esercitazioni sui problemi. Tramite
la definizione di questa carriera, realizzata per mezzo delle funzionalità
che seguono, si raggiunge l'obiettivo dell'autovalutazione delle personali
capacità dell'utente.

\begin{funcreq}
\textbf{}
\end{funcreq}



\newpage
\section{Requisiti non funzionali}

\subsection*{Caratteristiche delle risorse}

\begin{nonfuncreq}
\label{formatoproblema}
\textbf{Struttura di un problema:}
Tutti i problemi forniti possiedono:
\begin{itemize}
    \item Un titolo.

    \item Un testo, scritto prevalentemente in linguaggio naturale,
    che descrive uno scenario che richiede di essere risolto per mezzo
    di un algoritmo. Eventuali immagini e proposizioni matematiche
    formali possono accompagnare i testi.

    \item Almeno un esempio di input insieme al relativo output che
    mostra il risultato atteso.
\end{itemize}
\end{nonfuncreq}

\begin{nonfuncreq}
\label{metadata}
\textbf{Metadati dei problemi:} Ogni problema è essenzialmente un'entità
caratterizzata dai seguenti dati descrittivi:
\begin{itemize}
    \item Nome identificativo: si tratta di una stringa alfanumerica, che
    riprende una o più parole chiave del titolo del problema di cui al
    RNF \ref{formatoproblema}.
    \item Difficoltà: ogni problema possiede un'etichetta (\textit{tag})
    associata che ne valuta la difficolta in modo indicativo: bassa 
    intermedia, alta.
\end{itemize}
\end{nonfuncreq}

\subsection{Caratteristiche di sistema}

\begin{nonfuncreq}
\label{scalabilita}
\textbf{Scalabilità:}
L'infrastruttura del servizio deve essere scalabile e aperta alle esigenze
derivanti dall'aumento di nuovi utenti. Questo requisito è motivato dalla
disponibilità online del servizio che verrà sviluppato. In particolare:
\begin{enumerate}
    \item Data l'eterogeneità di linguaggi di programmazione esistenti
    al momento della stesura di questo documento, è importante che il
    servizio sia in grado di accogliere con l'avanzare del tempo codici
    scritti in linguaggi differenti.
\end{enumerate}
\end{nonfuncreq}

\begin{nonfuncreq}
\textbf{Usabilità:}
Le funzionalità 
\end{nonfuncreq}

\begin{nonfuncreq}
\textbf{Lingua di sistema:}
Il servizio sarà erogato in lingua italiana. Altrettanto sarà fatto per i
testi dei problemi.
\end{nonfuncreq}

\begin{nonfuncreq}
\textbf{Prestazioni:}
\end{nonfuncreq}

\subsection{Privacy e sicurezza}

\begin{nonfuncreq}
\textbf{Privacy:}
Il servizio deve essere progettato e realizzato in ottemperanza delle
vigenti disposizioni di legge in materia di tutela della privacy e
trattamento dei dati:
\begin{enumerate}
    \item L'applicazione fornita dal servizio deve essere conforme
    al regolamento \href{https://www.garanteprivacy.it/documents/10160/0/Regolamento+UE+2016+679.+Arricchito+con+riferimenti+ai+Considerando+Aggiornato+alle+rettifiche+pubblicate+sulla+Gazzetta+Ufficiale++dell%27Unione+europea+127+del+23+maggio+2018}{\textcolor{blue}{\underbar{UE n.2016/679}}} per
    la protezione dei dati.
\end{enumerate}
\end{nonfuncreq}

\begin{nonfuncreq}
\textbf{Sicurezza:}

\end{nonfuncreq}

\newpage
\section{Design front-end}

\section{Design back-end}

\end{document}
