\documentclass[11pt, a4paper]{article}

%\usepackage[T1]{fontenc}
%\usepackage{fullpage}

\usepackage[utf8]{inputenc} % comment when using lualatex
\usepackage[italian]{babel} % lingua e a-capo-sillabato
\usepackage{graphicx}
\usepackage[hidelinks]{hyperref} % link di pagina
\usepackage[bottom]{footmisc} % note appiccicate al fondo della pagina
\usepackage{float} % per posizionamento immagini
\usepackage{amsthm}
\usepackage{fancyhdr}
\usepackage[font=small,labelfont=bf]{caption} % small font for caption (and bold Figure word)
\usepackage{tabularx} %tabelle
\usepackage[table]{xcolor} %colore caselle

\pagestyle{fancy}
\fancyhf{}% Clear header/footer
\fancyfoot[C]{\thepage} %add page number
\fancyhead[C]{\footnotesize\textit{Documento:} D5 \hfill SleepCode \hfill \textit{Versione:} 1.0}
\renewcommand{\headrule}{{\color{red!70}\rule{\textwidth}{2pt}}}
\setlength{\headheight}{22pt}

%\pagestyle{myheadings}
%\markright{John Smith\hfill On page styles\hfill}

\renewcommand\UrlFont{\color{blue}\rmfamily}

\theoremstyle{definition}

\newtheorem{funcreq}{RF} %% numerazione dei requisiti funzionali
\newtheorem{nonfuncreq}{RNF} %% requisiti non funzionali
\newtheorem{backend}{BE}
\newtheorem{frontend}{FE}

\title{Report Finale}

\author{Raffaele \textsc{Castagna}\\
Alberto \textsc{Rovesti}\\
Zeno \textsc{Saletti}}

\newcommand{\groupNumber}{G17}

% footnote without number
\newcommand\blfootnote[1]{%
  \begingroup
  \renewcommand\thefootnote{}\footnote{#1}%
  \addtocounter{footnote}{-1}%
  \endgroup
}

% —

% Web address for the project (if any)
% \newcommand{\homepage}{\url{https://www.}}

% data
\date{\today}

\makeatletter{}

% IL PREAMBOLO FINISCE QUI %%%%%%%%%%%%%%%%%%%%%%%%%%%%%%%%%%%%%%%%%%%%%%%%%%%%






\begin{document}

% La pagina di copertina si trova in un file .tex a parte
% NON MODIFICARE QUESTO COMANDO!!!
\begin{titlepage}
\newcommand{\HRule}{\rule{\linewidth}{0.3mm}} % Defines a new command for horizontal lines, change thickness here
\center % Centre everything on the page

%------------------------------------------------
%	Logo
%------------------------------------------------
\includegraphics[width=0.3\textwidth]{materiale/UniTrento_logo_ITA_colore.png}\\[0.5cm]
%------------------------------------------------
%	Headings
%------------------------------------------------
\textsc{\Large Dipartimento di Ingegneria\\e Scienza dell'Informazione}\\[1.5cm]

{\Huge\textbf{Sleep Code}}\\[0.5cm]
\textsc{\large Progetto per il Corso di Ingegneria del Software}\\
\textsc{\large Anno Accademico 2023-2024}\\[0.5cm]

%------------------------------------------------
%	Title
%------------------------------------------------

\HRule\\[0.4cm]
{\huge\bfseries \@title}\\[0.1cm]
\HRule\\[1cm]

\begin{minipage}{\textwidth}
\begin{flushleft}
\textit{Descrizione:} documento di analisi dei requisiti funzionali, non funzionali, front-end e back-end.
\end{flushleft}
\end{minipage}\\[1.5cm]


\begin{minipage}{0.4\textwidth}
\begin{flushleft}
\large
\textit{Numero documento:} D1
\end{flushleft}
\end{minipage}
\begin{minipage}{0.4\textwidth}
\begin{flushright}
\large
\textit{Versione documento:} 2.4
\end{flushright}
\end{minipage}\\[1.5cm]

%------------------------------------------------
%	Author(s)
%------------------------------------------------
\begin{minipage}{0.4\textwidth}
\begin{flushleft}
\large
\textit{Membri del gruppo:}\\
\@author % Your name
\end{flushleft}
\end{minipage}
~
\begin{minipage}{0.4\textwidth}
\begin{flushright}
\large
\textit{Numero gruppo: }
\groupNumber
\end{flushright}
\end{minipage}

% 	If you don't want a supervisor, uncomment the two lines below and comment the code above
% 	{\large\textit{Author(s)}}\\
% 	\@author % Your name

%------------------------------------------------
%	Date
%------------------------------------------------

\vfill\vfill
\textit{Ultima revisione:}
{\@date}

\end{titlepage}

\tableofcontents\blfootnote{\textbf{Consigli utili per la consultazione del testo:} Se il lettore per file \texttt{.pdf} attualmente in uso lo consente, è possible navigare con più semplicità e velocità all'interno di questo documento cliccando sugli elementi dell'indice.}


\newpage
\section*{Scopo del documento}
Questo documento rappresenta il report finale del progetto. L'attenzione è
rivolta all'impegno dedicato alla realizzazione di tutti i
deliverables e del prototipo finale, l'organizzazione del lavoro e la sua
distribuzione tra i membri del gruppo (specificando non solo punti di forza
ma anche la consapevolezza delle criticità incontrate nel percorso), nonché
alla rilevanza delle nozioni relative all'ingegneria del software acquisite,
e applicate  questo progetto, durante il corso e i seminari.


\newpage
\section{Approcci all'ingegneria del software}
Nella presente sezione vengono riassunti brevemente i punti più significativi
dei seminari tenutisi durante il corso, ponendo particolare attenzione ai
metodi e ai principi dell'ingegneria del software emersi e descritti dai
relatori.

\subsection{BlueTensor}
\subsection{Il metodo Kanban}
\subsection{IBM}
\subsection{Meta}
\subsection{U-Hopper}
\subsection{Red Hat}
\subsection{Microsoft}
\subsection{Sistemi Legacy}
\subsection{Gestione del ciclo di vita del software e modernizzazione}
\subsection{APSS e tecnologie per servizi pubblici}

\newpage
\section{Organizzazione del lavoro}
\subsection{Ruoli e attività}
\begin{center}
  \footnotesize
  \begin{tabularx}{\textwidth}{|c||c||X|}
      \hline
      \cellcolor{red!70}Componente del team & \cellcolor{red!70}Ruoli & \cellcolor{red!70}Attività principali\\
      \hline
      Raffaele Castagna & Lorem ipsum dolor & Lorem ipsum dolor sit amet lorem ipsum dolor sit amet lorem ipsum dolor sit amet lorem ipsum dolor sit amet lorem ipsum dolor sit amet lorem ipsum dolor sit amet\\
      \hline
      Alberto Rovesti & Lorem ipsum dolor &\\
      \hline
      %redatt prjld
      Zeno Saletti & boh  &\\
      \hline
  \end{tabularx}
\end{center}

\subsection{Distribuzione del carico di lavoro}
\begin{center}
  \footnotesize
  \begin{tabularx}{\textwidth}{|X||c||c||c||c||c||c|}
      \hline
      \cellcolor{red!70}Componente del team & \cellcolor{red!70}D1 & \cellcolor{red!70}D2 & \cellcolor{red!70}D3 & \cellcolor{red!70}D4 & \cellcolor{red!70}D5 & \cellcolor{red!70}Totale\\
      \hline
      Raffaele Castagna & 11111&11111&111111&11111&1111&\\
      \hline
      Alberto Rovesti & &&&&&\\
      \hline
      Zeno Saletti & &&&&&\\
      \hline
      \cellcolor{red!70}Totale & &&&&&\\
      \hline
  \end{tabularx}
\end{center}

\section{Criticità}
\section{Autovalutazione}
\begin{center}
  \footnotesize
  \begin{tabularx}{\columnwidth}{|X||c|}
      \hline
      \cellcolor{red!70}Componente del team & \cellcolor{red!70}Voto\\
      \hline
      Raffaele Castagna & \\
      \hline
      Alberto Rovesti & \\
      \hline
      Zeno Saletti & \\
      \hline
  \end{tabularx}
\end{center}

\end{document}